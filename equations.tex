% !TEX root = ./notes.tex
\chapter[Stellar Structure Equations]{The Lagrangian Equations of Stellar Structure}

\section{The conservation laws}

After our rapid overview, we now gather the tools needed to tackle stellar evolution.  The first is to get the macroscopic equations for stellar structure. We will start from the equations expressing conservation of mass\footnote{In a relativistic system, we would instead start from conservation of baryon number.}, momentum, and energy. We already derived, in \S\ref{s.fluid-introduction}, the continuity (conservation of mass) equation,
\begin{equation}\label{e.mass-1}
\partial_{t}\rho + \divr(\rho\vu) = 0,
\end{equation}
and the Euler equation,
\begin{equation}\label{e.momentum-1}
\partial_{t}\vu + \vu\cdot\grad\vu = -\grad \Phi - \frac{1}{\rho}\grad P.
\end{equation}
Note that if we multiply eq.~(\ref{e.momentum-1}) by $\rho$, we can rewrite it, using eq.~(\ref{e.mass-1}), as
\begin{equation}\label{e.momentum-2}
	\partial_{t}(\rho\vu) + \divr[\vu(\rho\vu)] = -\rho\grad\Phi -\grad P.
\end{equation}
The left-hand side is interpreted as expressing the conservation of momentum ($\rho\vu$) in the absence of forces, analogous to eq.~(\ref{e.mass-1}) for the conservation of mass ($\rho$).

Note the general form of a conservation equation:
\begin{eqnarray*}
\lefteqn{\partial_{t}(\textrm{conserved quantity})}\\
 & & \mbox{} + \divr(\textrm{flux of conserved quantity}) =
 (\textrm{sources}) - (\textrm{sinks}).
\end{eqnarray*}
Because the momentum density $\rho\bvec{u}$ is a vector, its flux is a tensor: $[\bvec{u}(\rho\bvec{u})]_{ij} \equiv \rho u_{i}u_{j}$.

The next equation is that of energy conservation. Here we must consider both the internal energy per unit volume $E/V = \rho \varepsilon$ and the kinetic energy per unit volume $\rho u^{2}/2$.  In this section $\varepsilon$ represents the internal energy per unit mass of the fluid. In a fixed volume of the fluid the total energy is then 
\[ \int_{V}(\rho \frac{1}{2}u^{2} + \rho \varepsilon)\,\dif V. \]
The flux of energy into this volume will clearly include
\[ -\int_{V}\left(\frac{1}{2}\rho u^{2} + \rho \varepsilon\right) \vu\vdot\dif\bvec{S}. \]
But wait, there's more!  In addition, we have a conductive heat flux, 
\[-\int_{\partial V}\bvec{F}\vdot\dif\bvec{S}.\] 
Moreover, the pressure acting on fluid flowing into our volume does work on the gas at a rate 
\[-\int_{\partial V}P\vu\vdot\dif\bvec{S}.\] 
As a result, the net change of energy in our volume is
\begin{eqnarray}
\lefteqn{\partial_{t}\int_{V}\left(\frac{1}{2}\rho u^{2} + \rho \varepsilon\right)\nsp\dif V = }\nonumber \\
 && -\int_{\partial V} \!\dif\bvec{S}\vdot\left[\vu\left(\frac{1}{2}\rho u^{2} + \rho \varepsilon + P\right) + \bvec{F}\right] + \int_{V} \left( \rho \vu\vdot\bvec{g} + \rho q\right).
\label{e.energy-1}
\end{eqnarray}
On the right-hand side we've added in the work done by gravity and the heating evolved by nuclear reactions (this could also involve sinks, such as neutrinos, which have a long mean free path).
Expressed in differential form, this is
\begin{equation}\label{e.energy-2}
 \partial_{t}\left(\frac{1}{2}\rho u^{2} + \rho \varepsilon\right) 
 	+ \divr\left[\rho\vu\left(\frac{1}{2} u^{2} + \varepsilon + \frac{P}{\rho}\right)\right]
	+ \divr\bvec{F} = \rho q + \rho \vu\vdot\bvec{g}.
\end{equation}
You are possibly wondering why I didn't put gravity, which can be expressed as a potential, on the left hand side of this equation.  The reason is that the gravitational stresses cannot be expressed in a  \emph{locally} conservative form; it is only when integrating over all space that the conservation law appears.

Equations~(\ref{e.mass-1}), (\ref{e.momentum-2}), and (\ref{e.energy-2}) are supplemented by an equation of state, which allows one to get from $P$ and $T$, plus the composition (specified by mass fractions $X_{i}$), the remaining quantities $\rho$ and $\varepsilon$. In addition, Poisson's equation
\begin{equation}\label{e.poisson}
\nabla^{2}\Phi = 4\pi G\rho,
\end{equation}
specifies the gravitational acceleration $\bvec{g} = -grad\Phi$. We then need one more equation to specify the heat flux $F$. We argued in \S\ref{s.energy-transport-estimate} that the typical length over which a photon travels before scattering is very small compared to the lengthscale over which the macroscopic properties of the star vary.  In this case, we expect the flux to obey a conduction equation of the form
\begin{equation}\label{e.conduction-simple}
\bvec{F} = -K\grad T.
\end{equation}
This assumption is clearly questionable near the stellar surface, and we have left unspecified the form of $K$.  Such an equation does, however, close the system of equations; all of the physics is then contained in the equation of state $P(\rho,T,\{X_{i}\})$, the rate of heating from nuclear reactions $q(\rho, T, \{X_{i}\})$, and the thermal conductivity $K(\rho,T,\{X_{i}\})$.  We will also need a system of equations to describe how the $X_{i}$ change as a result of the nuclear reactions.

\section{The equations in Lagrangian form}

Because the mass is constant, It is more useful to transform coordinates from $(r,t)$ to $(m,t)$, where $m$ is the mass within a sphere of radius $r$.  

It is often desirable to use the mass enclosed by a surface of radius $r$,
\begin{equation}\label{e.mass}
	m(r,t) = \int_{0}^{r}\! \rho(r',t) 4\pi r'^{2} \,\dif r',
\end{equation}
as a Lagrangian coordinate.
To do this, differentiate eq.~(\ref{e.mass}) w.r.t.\ $r$,
\[ \partial_{r}m = 4\pi r^{2}\rho, \]
and substitute for $\rho$ in the equation of  continuity (eq.~[\ref{e.mass-conv}]).  The first term becomes
\[ 
	\partial_{t}\rho = \partial_{t}\left(\frac{1}{4\pi r^{2}} \partial_{r} m\right) 
	= \frac{1}{4\pi r^{2}}\partial_{r}(\partial_{t}m),
\]
while the second term becomes
\[
	\frac{1}{4\pi r^{2}}\partial_{r}\left(u\partial_{r}m\right);
\]
the equation of continuity therefore becomes
\begin{equation}\label{e.mod-continuity}
	\frac{1}{4\pi r^{2}} \partial_{r}\left( \partial_{t} m + u\partial_{r} m\right) = 0.
\end{equation}
We can integrate this over $r$ to find that $\partial_{t} m + u\partial_{r} m = f(t)$; since $m(0,t) = 0,\;\forall t$, we must have $f(t) = 0$.  Now $\partial_{t} m + u\partial_{r} m = \dif m/\dif t = 0$, so along a streamline, $m$ is a constant.  We can therefore transform from coordinates $(r,t)$ to $(m,t)$ by setting
\begin{eqnarray}
	\label{e.lagrange-rule-1}
	\left.\frac{\partial}{\partial_{t}}\right|_{r} + u\left.\frac{\partial}{\partial r}\right|_{t} 
	&=& \left.\frac{\partial}{\partial t}\right|_{m} \equiv \frac{\Dif}{\Dif t}\\
	\label{e.lagrange-rule-2}
	\left.\frac{\partial}{\partial r}\right|_{t} &=& 4\pi r^{2}\rho \left.\frac{\partial}{\partial m}\right|_{t}.
\end{eqnarray}
Here $\Dif/\Dif t \equiv (\partial/\partial t)_{m}$ is the Lagrangian time derivative.  In deriving this change, we used the equation of continuity, which becomes
\begin{equation}\label{e.lagrange-r}
\frac{\partial r}{\partial m} = \frac{1}{4\pi r^{2}\rho}.
\end{equation}
Our equation for momentum (eq.~[\ref{e.momentum-1}]) becomes
\begin{equation}\label{e.lagrange-momentum}
\frac{\partial P}{\partial m} = -\frac{Gm}{4\pi r^{4}} - \frac{1}{4\pi r^{2}}\frac{\Dif u}{\Dif t}.
\end{equation}
In hydrostatic balance the second term on the right-hand side is negligible.  
The flux equation, (eq.~[\ref{e.conduction-simple}]) can be transformed to
\begin{equation}\label{e.lagrange-flux}
\frac{\partial T}{\partial m} = - \frac{1}{16\pi^{2} r^{4}\rho K}L_{r}
\end{equation}
Here $L_{r}$ is the luminous flux at a radius $r$. 
%	\frac{\partial T}{\partial m} = -\frac{3}{64\pi^{2}r^{4}}\frac{\kappa}{ac T^{3}}L_{r}.

The energy equation (eq.~[\ref{e.energy-2}]) is more complicated. We can expand the time derivative as
\begin{eqnarray*}
	\partial_{t}(\frac{1}{2}\rho u^{2} + \rho \varepsilon) 
	&=& \left(\frac{1}{2}u^{2} + \varepsilon\right)\partial_{t}\rho + \rho\partial_{t}\left[\frac{1}{2}(\vu\vdot\vu) + \varepsilon\right]\\
	&=& -\left(\frac{1}{2}u^{2} + \varepsilon\right)\divr\left(\rho\vu\right) + \rho \vu\partial_{t}\vu + \rho\partial_{t}\varepsilon,
\end{eqnarray*}
using equation~(\ref{e.mass-1}) to substitute for $\partial_{t}\rho$.  We then use equation~(\ref{e.momentum-1}) to replace $\partial_{t}\vu$, and recognizing that $\vu(\vu\vdot\grad)\vu = \vu\vdot\grad[(1/2)u^{2}]$, rewrite equation~(\ref{e.energy-2}) as
\[ 
	\rho\left(\partial_{t} + \vu\vdot\grad\right) \varepsilon + P\divr\vu = -\divr\bvec{F} + \rho q.
\]
We've canceled all common factors here.  Finally, we once again use equation~(\ref{e.mass-1}) to set 
\[
	P\divr\vu = -(P/\rho)(\partial_{t}\rho + \vu\vdot\grad \rho) 
	= \rho P\left(\partial_{t} + \vu\vdot\grad\right)\left(\frac{1}{\rho}\right).
\]
Now we have the left side in terms of the Lagrangian time derivative $\Dif/\Dif t$.  Note that when we combine terms, the left-hand side contains 
\[ \frac{\Dif \varepsilon}{\Dif t} + P\frac{\Dif (1/\rho)}{\Dif t} = T\frac{\Dif s}{\Dif t} \] 
where the equality follows from the first law of thermodynamics.

For the right-hand side, we expand the divergence operator in spherical symmetry and use equation~(\ref{e.lagrange-rule-2}) to obtain
\[
	-\divr\bvec{F} = -\frac{1}{r^{2}}\frac{\partial(r^{2} F)}{\partial r} = -\rho\frac{\partial L_{r}}{\partial m}.
\]
Putting everything together, we finally have our Lagrangian heat equation,
\begin{equation}\label{e.lagrange-heat}
	\frac{\partial L_{r}}{\partial m} = q - T\frac{\Dif s}{\Dif t}.
\end{equation}
This has a simple interpretation: the change in luminosity across a mass shell is due to sources or sinks of energy and the change in the heat content of the shell.  It is more useful, however, to work with temperature and pressure instead of entropy.  Write
\[
	T\frac{\Dif s}{\Dif t} = T\tderiv{s}{T}{P}\frac{\Dif T}{\Dif t} + T\tderiv{s}{P}{T}\frac{\Dif P}{\Dif t},
\]
and use the identity (see Appendix~\ref{s.thermo-derivatives})
\[
	\tderiv{s}{P}{T} = -\tderiv{s}{T}{P}\tderiv{T}{P}{s}
\]
to obtain
\begin{equation}\label{e.lagrange-heat-alt}
	\frac{\partial L_{r}}{\partial m} 
	= q - c_{P}\left[ \frac{\Dif T}{\Dif t} - \tderiv{T}{P}{s} \frac{\Dif P}{\Dif t}\right].
\end{equation}
Equations~(\ref{e.lagrange-r}), (\ref{e.lagrange-momentum}), (\ref{e.lagrange-flux}), and (\ref{e.lagrange-heat-alt}), when supplemented by an equation of state, a Rosseland mean opacity, and the equations for nuclear heating and neutrino cooling, are the equations for stellar structure and evolution in spherical symmetry. We must also include a treatment of the photosphere (chapter~\ref{s.stellar-atmospheres}) and a phenomenological treatment of convection as well.
