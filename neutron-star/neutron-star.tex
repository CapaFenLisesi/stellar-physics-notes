% !TEX root = ../stellar-notes.tex
\chapter{Neutron Stars}

\section{Cooling neutron stars}\label{s.cooling-neutron-stars}
First, let's get some estimates of the neutron star.  A solar mass has of order $10^{57}$ nucleons.  If we pack those nucleons so that the mean spacing is $\approx 1\nsp\fermi$, then the radius of our object is $\sim 10^{19}\nsp\fermi\sim 10\nsp\km$.  The current best observational constraints put the radius at around \val{12}{\km}.  Observed masses (from radio pulsars in binaries) range from around \val{1.1}{\Msun} up to \val{1.97}{\Msun}.

Although the matter in the core of a neutron star is not an ideal gas, we'll start by ignoring the nuclear force and imagining that the core consists of cold  neutrons, proton, and electrons in a charge-neutral plasma that is in $\beta$-equilibrium. This translates into three conditions:
\begin{enumerate}
\item\label{cold-neutron-star} Cold: The chemical potentials are just the Fermi energies, $\mu = E_{\mathrm{F}}$, with
\begin{eqnarray}
	E_{\mathrm{F}} &=& \frac{\hbar^{2}}{2m}\left(3\pi^{2}n\right)^{2/3},\qquad\textrm{non-relativistic}\\
	E_{\mathrm{F}} &=& \hbar c\left(3\pi^{2}n\right)^{1/3},\qquad\textrm{relativistic}.
\end{eqnarray}
\item\label{charge-neutral-star} Charge-neutral: $n_{e} = n_{p}$.
\item\label{beta-equil-star} Beta-equilibrium: The reaction $\nt\leftrightarrow \pt+e$ is in equilibrium, so 
$ \mu_{n} = \mu_{p}+\mu_{e} - Q$,
where $Q = (m_{n}-m_{p}-m_{e})c^{2} = 0.782\nsp\MeV$.
\end{enumerate}

The proton fraction (and therefore the electron fraction) is indeed quite small at low density (near nuclear saturation density) where the protons and neutrons are non-relativistic (exercise).  The proton-to-neutron ratio increases with density.  The reactions to maintain $\beta$-equilibrium emit a neutrino: $\nt \to \pt + e + \bar{nu}_{e}$ and $\pt + e \to \nt + \nu_{e}$.  After the first few seconds, the temperatures are well below $10\nsp\MeV$, and the neutrino mean free path is larger than the stellar radius.  Since the star is degenerate, the source of free energy for the neutrinos cannot come from contraction, and the neutron star must cool.

\begin{exercisebox}
 Find the equilibrium proton-to-neutron ratio, $x = n_{p}/n_{n}$.  The electrons, being light, are always relativistic.  Show that $x$ is indeed vary small but increases with density in the limit of both neutrons and protons being non-relativistic. In the opposite limit, all three particles relativistic, show that $x\to 1/8$.  \emph{Hint:} neglect the electron mass and the difference between the neutron and proton masses; use the values $\hbar c = 197.327\nsp\MeV\usp\fermi$ and $m_{n}c^{2} = 939.565\nsp\MeV$, and scale the neutron density $n_{n}$ to the nuclear saturation density $n_{0}= 0.16\nsp\fermi^{-3}$.
\end{exercisebox}

These reactions (named \emph{Urca} reactions by Gamow) would indeed cause the neutron star to cool exceedingly rapidly, but there is one hitch: the reactions are blocked!  As a result, neutrino emission proceeds via the modified Urca reactions:
\begin{eqnarray}
\nt+\nt &\to& \nt + \pt + e + \bar{\nu}_{e}\\
\nt + \pt + e &\to& \nt + \nt + \nu_{e}.
\end{eqnarray}

\begin{exercisebox}
Show that the reactions
\begin{eqnarray*}
\nt &\to& \pt + e + \bar{\nu}_{e}\\
\pt +e &\to& \nt + \nu_{e}
\end{eqnarray*}
cannot simultaneously conserve momentum and energy if the $\nt$, $\pt$, and $e$ are on their respective Fermi surfaces.  Take the neutrons and protons to be non-relativistic, the electrons to be relativistic, and presume that the plasma is charge-neutral and in $\beta$-equilibrium. Neglect the electron rest mass and the difference between the neutron and proton rest masses.  Recall that the momentum of a particle on its Fermi surface is $p_{\mathrm{F}} = \hbar(3\pi^{2} n)^{1/3}$, so that $E_{\mathrm{F}} = p_{\mathrm{F}}^{2}/(2m)$ if non-relativistic and $E_{\mathrm{F}} = p_{\mathrm{F}}c$ if relativistic.
\end{exercisebox}


\section[Accreting Neutron Star Envelopes]{Structure of an Accreting Neutron Star Envelope}

This section is similar in spirit to lectures\cite{bildsten:thermonuclear}  given at a NATO Advanced Studies Institute by Prof.~Bildsten, KITP.

As a worked example, we shall consider the thermonuclear stability of hydrogen/helium accretion onto a neutron star.  For accretion from a disk, the luminosity (cf.\ problem~\ref{p.disk-L}) is $GM\dot{M}/(2R)$.  The remainder of the gravitational binding energy is liberated at a boundary layer where the disk becomes incorporated into the star.  Setting this equal to thermal emission gives an effective temperature
\[
	\Teff = \left(\frac{GM\dot{M}}{2R\ssb\mathcal{A}}\right)^{1/4} = 10^{7}\nsp\K \left(\frac{M}{1.4\nsp\Msun}\frac{10^{6}\nsp\cm}{R}\frac{10^{13}\nsp\cm^{2}}{\mathcal{A}}\frac{\dot{M}}{10^{-9}\nsp\Msun\usp\yr^{-1}}\right)^{1/4}
\]
Here we have scaled the emitting area to $\mathcal{A}$ rather than presuming it is emitted from the entire surface. This effective temperature is, in energy units, $\kB \Teff\approx \nsp\keV$, which is in the X-ray region of the spectrum.  

The gravitational acceleration at the neutron star surface is
\[
	g = \frac{GM}{R^{2}}\left(1-\frac{2GM}{Rc^{2}}\right)^{-1/2}\approx 2.4\ee{14}\nsp\cm\usp\second^{-2}
\]
for a neutron star with $M = 1.4\nsp\Msun$, $R = 10^{6}\nsp\cm$.  The factor in parenthesis is a correction due to general relativity. The redshift, i.e., the fractional increase in wavelength of a photon emitted from the surface to a distant observer, is $\Delta \lambda/\lambda = [1-2GM/(Rc^{2}]^{-1/2} - 1 \approx 0.3$.   Given the surface gravity and the temperature, we can estimate the pressure scale height,
\[
 H = \frac{P}{\rho g} = \frac{\kB T}{\mu\mb g} \approx 5\nsp\cm.
\]
Because $H/R\ll 1$, we can make two simplifying assumptions: first, we can treat our atmosphere in planar geometry, and second we can boost to a local frame and perform our calculations in Euclidian space-time.

In a planar geometry, there is a simple relation between the time since a fluid element was deposited and its local pressure. If we assume that accreted matter spreads uniformly over the surface, than every square centimeter sees mass being added at a rate $\dot{m} = \dot{M}/(4\pi R^{2})$.  Thus after a time $t$ a fluid element will find itself under a column with a mass per unit area $\dot{m}\times t = \int_{z}^{\infty}\rho\usp\dif z$. Using hydrostatic equilibrium, we can transform $\rho\usp\dif z = -\dif P/g$, so that
\[
P(t) = \dot{m}tg.
\]
Here $t$ is the time since the fluid element was accreted.  At an accretion rate of $\dot{M} = 10^{-9}\nsp\Msun\usp\yr^{-1}$ onto a $10\nsp\km$ neutron star, the pressure reached when the hydrogen is half-consumed is $P = 1.8\ee{22}\nsp\dyne\usp\cm^{-2}$.  The quantity $y\equiv \int_{z}^{\infty}\rho\usp\dif z$ is known as the \emph{column density}.

\paragraph{Hydrogen burning}
Let's consider the accretion of a (roughly solar) mixture of \hydrogen, \helium, and \carbon, with mass fractions $X_{\mathrm{H}}:X_{\mathrm{He}}:X_{\mathrm{CNO}} = 0.70:0.28:0.02$. At temperatures $> 10^{7}\nsp\K$, the \hydrogen\ is consumed via the hot CNO cycle,
\[
	\carbon(\pt,\gamma)\nitrogen[13](\pt,\gamma)\oxygen[14](\beta^{+}\nu)\nitrogen[14](\pt,\gamma) \oxygen[15](\beta^{+}\nu)\nitrogen[15](\pt,\alpha)\carbon.
\]
The two slowest  reactions are the decays of \oxygen[14] ($\tau_{1/2} = 71\nsp\second$) and \oxygen[15] ($\tau_{1/2} = 122\nsp\second$). As a result, the cycle is independent of temperature and thermally stable.
How long would it take to consume all of the hydrogen?  For each CNO nucleus, there are
\[ n_{\mathrm{H}}/n_{\mathrm{CNO}} = \frac{0.7}{0.02/12} = 420 \]
hydrogen nuclei, and each trip around the cycle consumes 4 hydrogen nuclei. As a result, it take 105 trips around the CNO cycle to deplete the hydrogen.  The time to complete one cycle is $(71\nsp\second + 122\nsp\second) /\ln 2 = 278\nsp\second$, so the time to deplete all of the hydrogen is $2.92\ee{4}\nsp\second$---about $1/3$ of a day.

To get the temperature structure of the accreted envelope, we use the flux equation, eq.~(\ref{e.flux}),
\begin{equation}\label{e.mod-flux}
F = -\frac{1}{3}\frac{c}{\rho\kappa}\frac{\dif aT^{4}}{\dif r}.
\end{equation}
We can again transform $\rho\usp\dif r = -\dif P/g$; moreover, if all of the heat released by fusing hydrogen to helium comes out (and assuming there is no other source of heat in the neutron star interior), the flux will just be $QX_{\mathrm{H}}\dot{m}$ and will be constant. Integrating eq.~(\ref{e.mod-flux}) and neglecting $\Teff$ and $P(\textrm{photosphere})$ gives us a relation for the temperature,
\[	T^{4} = \frac{3\kappa Fy}{ac}. \]
For the parameters we are considering, $F = QX_{\mathrm{H}}\dot{m} = 2.1\ee{22}\nsp\ergspersecond\usp\cm^{-2}$, and the temperature at the depth where the H is half-consumed is $T = 2.9\ee{8}\nsp\K$.

The cooling timescale for this atmosphere is
\[ t_{\mathrm{th}} = \frac{yC_{P} T}{F} \approx 208\nsp\second \times \frac{y_{8} T_{8}}{F_{22}} \]
Here we are using the shorthand notation, $y_{8}\equiv y/(10^{8}\nsp\columnunit)$, and similarly for $T_{8}$, $F_{22}$, and so on.
The thermal timescale can be compared to the accretion timescale,
\[ t_{\mathrm{ac}} = \frac{y}{\dot{m}} = 10^{4}\nsp\second\times y_{8}\dot{m}_{4}^{-1}.
\]
Because $t_{\mathrm{th}}\ll t_{\mathrm{ac}}$, the compression is slow and far from adiabatic.

As the fluid element is compressed and heated, the lifetime of a \helium\ nucleus against the triple-alpha reaction becomes shorter than the accretion timescale and helium ignites.  If protons are present, then there will be additional heating from $\carbon(\pt,\gamma)$.  The triple-alpha reaction is strongly temperature sensitive and therefore is susceptible to an thermal instability.

\paragraph{Helium ignition}
To evaluate the thermal instability, we approximate the cooling rate as
\begin{equation}\label{e.cooling-rate-approx}
\epsilon_{\mathrm{cool}} \approx \frac{C_{P}T}{t_{\mathrm{th}}} = \frac{ac T^{4}}{3\kappa y^{2}}
\end{equation}
and compare it against the heating rate from the triple-alpha reaction, $\epsilon_{3\alpha}$. We then apply a constant-pressure thermal perturbation about the steady-state, $\epsilon_{\mathrm{cool}} = \epsilon_{3\alpha}$; upon expanding this equation we arrive at a condition for instability:
\begin{equation}\label{e.3a-instability}
\left.\frac{\partial\ln\epsilon_{3\alpha}}{\partial\ln T}\right|_{y} > \left.\frac{\partial\ln\epsilon_{\mathrm{cool}}}{\partial\ln T}\right|_{y}.
\end{equation}
The triple-alpha heating rate is 
\[
	\epsilon_{3\alpha} \approx (5.3\ee{21}\nsp\ergspersecond\usp\gram^{-1}) \times \frac{\rho_{5}^{2}X_{\mathrm{He}}^{3}}{T_{8}^{3}}\exp\left(-\frac{44}{T_{8}}\right).
\]
If we ignore the variation in density, $\rho = 10^{5}\nsp\grampercc \times \rho_{5}$, with temperature, then equation~(\ref{e.3a-instability} becomes
\begin{equation}\label{e.3a-instability-temperature}
\frac{44}{T_{8}} - 3 > 4;
\end{equation}
put another way, a thermal instability is expected to occur if $T < 6.3\ee{8}\nsp\K$ at the depth where helium is consumed.  This is satisfied for all $\dot{m}< \dot{m}_{\mathrm{Edd}}$.

The remaining question is the depth at which the triple-alpha reaction ignites.  For a rapidly accreting neutron star, the flux is greater, and therefore the envelope has a steeper temperature gradient\cite{bildsten:thermonuclear}; roughly, for $\dot{m} \gtrsim 0.1\dot{m}_{\mathrm{Edd}}$, the helium ignition occurs in a bath of protons, so that an rp-process ensues. The consumption of hydrogen releases more energy and requires waiting for $\beta+$-decays, this tends to produce a longer-lasting X-ray burst that is more energetic, for the amount of fuel accreted.    At lower accretion rates, the hot CNO cycle consumes all of the available hydrogen before the helium ignites; as a result, the burst is more rapid, and less energetic for the amount of matter accreted.  Note that the pure helium burst may still release a greater total amount of energy if it ignites at a greater depth, i.e., if the amount of matter accumulated between bursts is greater.
