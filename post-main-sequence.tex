% !TEX root = ./stellar-notes.tex
\chapter[Low-Mass Post-Main Sequence]{Post-Main Sequence Evolution: Low-mass Stars}

\section{The Triple-Alpha Reaction}\label{s.triple-alpha}

The consumption of \helium\ is hindered by the lack of stable nuclei with mass numbers $A=5$ and $A=8$.  
The nucleus \beryllium[8] is, however, long-lived by nuclear standards: its decay width is $\Gamma = 68\nsp\eV$, implying a decay timescale $\hbar/\Gamma = 9.7\ee{-17}\nsp\second$.  (For comparison, the lifetime of \lithium[5] is $\sim 10^{-22}\nsp\second$.)  Thus if the reaction $\helium[4]+\helium[4]\to\beryllium[8]$ can proceed quickly enough, a small amount of $\beryllium[8]$ can accumulate allowing the reaction $\beryllium[8]+\helium\to\carbon$ to proceed.

The reaction $2\nsp\helium\to\beryllium[8]$ is endothermic, with $Q = -92\nsp\keV$.  As a result, the peak energy for Coulomb barrier transmission (see the discussion following eq.~[\ref{e.integral}]) must reach
\[ \Epk = \frac{\EG^{1/3}(kT)^{2/3}}{4^{1/3}} = -Q. \]
Substituting $\EG = 979\nsp\keV A (Z_{1}Z_{2})^{2}$ and solving for $T$ gives $T = 1.2\ee{8}\nsp\K$ as the temperature required to build up any substantial amount of \beryllium[8].  To reach such a temperature requires a helium core mass of $\gtrsim 0.45\nsp\Msun$.

Once a sufficient temperature is reached, the reaction $2\nsp\helium[4]\to\beryllium[8]$ comes into equilibrium with the decay, $\beryllium[8]\to2\nsp\helium$.  We can use a Saha-like equation (cf.\ eqns.~[\ref{e.ionization-chem-potential}] and [\ref{e.saha-eqn}]) to get the abundance of \beryllium[8].  Writing
\[ 2\mu_{4} = \mu_{8} - Q \]
and substituting the expression for $\mu$, eq.~(\ref{e.chem-pot-ideal-gas}) gives
\begin{equation}\label{e.n8-n4}
\frac{n_{8}}{n_{4}} = \left(\frac{2\pi\hbar^{2}}{\mb kT}\right)^{3/2}\left(\frac{8}{4^{2}}\right)^{3/2}\frac{X_{4}\rho}{4\mb}\exp\left(-\frac{92\nsp\keV}{kT}\right).
\end{equation}
Here we use $n_{8}$ and $n_{4}$ to mean the number densities of, respectively, \beryllium[8] and \helium; also $X_{4}$ denotes the mass fraction of helium.
Scaling eq.~(\ref{e.n8-n4}) to $\rho = \rho_{5} 10^{5}\nsp\grampercc$ and $T = T_{8}10^{8}\nsp\K$, we obtain
\begin{equation}\label{e.n8-to-n4}
 \frac{n_{8}}{n_{4}} = 2.8\ee{-5} T_{8}^{-3/2}X_{4}\rho_{5}\exp\left(-\frac{10.68}{T_{8}}\right).
\end{equation}
At $X_{4}\rho_{5}=1$ and $T_{8} = 1$, $n_{8}/n_{4} = 6.5\ee{-10}$.  

If the reaction $\beryllium[8]+\helium\to\carbon$ were non-resonant, the reaction rate for this $n_{8}$ would be far too slow to account for the amount of \carbon\ synthesized in stars.  Hoyle proposed, therefore, that there should be an excited state of the \carbon\ nucleus into which the reaction would proceed.  Both \helium\ and \beryllium[8] have spin and parity $J^{\pi} = 0^{+}$; hence for $s$-wave capture (angular momentum $\ell = 0$), the state in \carbon\ should also have $J^{\pi} = 0^{+}$.  What energy should the level have? The $Q$-value for $\beryllium[8]+\helium\to\carbon$ is $7.367\nsp\MeV$; the Gamow energy for this reaction is $\EG = 1.67\ee{5}\nsp\keV$ and hence the peak energy is $\Epk = 146\nsp\keV$, with a width $\Delta = 4(\Epk kT/3)^{1/2} = 82\nsp\keV$.  The proposed level should therefore have an energy within $2\Delta$ of $7.513\nsp\MeV$.

Such a level was indeed detected by Fowler, with $J^{\pi} = 0^{+}$ and $E = 7.654\nsp\MeV$.  Radiative decay from this level is hampered: the ground state also has $J^{\pi} = 0^{+}$, so the decay is forbidden; and the decay to the $J^{\pi}=2^{+}$ state at $4.44\nsp\MeV$ has a decay width of only $\Gamma_{\mathrm{rad}} = 3.67\milli\eV$.  This level's primary decay is indeed back to $\beryllium[8] + \helium$.  If the forward rate is fast enough, however, than a population of \carbon\ in this excited state can accumulate.  The total rate to the ground state would then be $n_{12*}\Gamma_{\mathrm{rad}}/\hbar$, where $n_{12*}$ is the number density of \carbon\ nuclei in the excited state.

To compute $n_{12*}$, we again can use the equation for chemical equilibrium: $\mu_{8} + \mu_{4} = \mu_{12*} - Q_{*}$.  Here $Q_{*} = -287\nsp\keV$ is the difference in energy between the \helium\ and \beryllium[8] nuclei and the energy of \carbon\ \emph{in the excited state.}  Again using eq.~(\ref{e.chem-pot-ideal-gas}) to expand $\mu$, we obtain
\begin{equation}\label{e.n12}
n_{12*} = \frac{n_{Q,12*}}{n_{Q,4}n_{Q,8}} n_{4}n_{8}\exp\left(-\frac{287\nsp\keV}{kT}\right).
\end{equation}
Substituting for $n_{8}$ using equation~(\ref{e.n8-n4}), this becomes
\begin{equation}\label{e.n12-2}
n_{12*} = \left(\frac{2\pi\hbar^{2}}{\mb kT}\right)^{3}\left(\frac{12}{4^{3}}\right)^{3/2} 
	\left(\frac{X_{4}\rho}{4\mb}\right)^{3} \exp\left(-\frac{397\nsp\keV}{kT}\right).
\end{equation}
Multiplying eq.~(\ref{e.n12-2}) by $\Gamma_{\mathrm{rad}}/\hbar$ and scaling to $\rho_{5}$, $T_{8}$, we obtain the net rate, per unit volume, at which $3\nsp\helium\to\carbon$,
\begin{equation}\label{e.triple-alpha}
 \textrm{rate} = 4.37\ee{31}\nsp\cm^{-3}\usp\second^{-1}\frac{(X_{4}\rho_{5})^{3}}{T_{8}^{3}}\exp\left(-\frac{44.0}{T_{8}}\right).
\end{equation}
Multiplying the rate by $Q = 7.275\nsp\MeV$, the net $Q$-value, gives the volumetric heating rate $\rho\varepsilon_{3\alpha}$, or
\begin{equation}\label{e.triple-alpha-heating}
\varepsilon_{3\alpha} \approx 5.1\ee{21}\nsp\ergs\usp\gram^{-1}\usp\second^{-1}\frac{X_{4}^{3}\rho_{5}^{2}}{T_{8}
^{3}}\exp\left(-\frac{44.0}{T_{8}}\right).
\end{equation}
The temperature exponent is $\dif\ln\varepsilon_{3\alpha}/\dif\ln T = 44.0/T_{8}-3$, that is, $\varepsilon_{3\alpha}\sim T^{41}$ at $T = 10^{8}\nsp\K$.

\subsection[Core helium burning]{The helium flash and the horizontal branch}\label{s.burning-stability}

The extreme temperature sensitivity of the triple-alpha reaction motivates a return to analyzing the stability of reactions in a stellar environment. We saw in an earlier problem that the ``gravithermal'' specific heat 
\[ C_{\star} \equiv T\left.\frac{\partial S}{\partial T}\right|_{M} < 0 \]
for an ideal gas. The physical cause is that an increase in entropy leads to an increase in radius, and the resulting $P\,\dif V$ work results in a reduced central temperature.

For conditions in low-mass stars at the time of helium ignition (nearly pure He at $T \approx 10^{8}\nsp\K$, $\rho \approx 10^{5}\nsp\grampercc$); at that density the temperature at $\kB T = \eF$ is (eq.~[\ref{e.TF}]) $4.1\ee{8}\nsp\K$.  Thus, He ignition takes place under semi-degenerate conditions.  To understand how the star responds, let's assume a homologous expansion---that is, one in which the ratios $r(m)/R(M)$ remains constant.  In other words, we assume that the structure of the star retains its functional form and we are merely ``rescaling'' our radial length.  To compute a gravithermal specific heat, we use Jacobians (see Appendix~\ref{s.thermo-derivatives}) to transform $S(T,P)$ to $S(T,M)$:
\begin{eqnarray}\label{e.homologous-cstar}
T\tderiv{S}{T}{M} &=& T\jac{S}{M}{T}{M}\nonumber \\
 &=& T\jac{S}{M}{T}{P}\jac{T}{P}{T}{M}\nonumber \\
 &=& T\tderiv{S}{T}{P} - T\tderiv{S}{P}{T}\tderiv{M}{T}{P}\tderiv{P}{M}{T}\nonumber \\
 &=& C_{P}\left[ 1 - \tderiv{T}{P}{S}\tderiv{P}{T}{M}\right],
\end{eqnarray}
where in the last identity we have used 
\[
 	\left(\frac{\partial T}{\partial P}\right)_{S} 
 	\left(\frac{\partial S}{\partial T}\right)_{P} 
 	\left(\frac{\partial P}{\partial S}\right)_{T} = -1
\]
and a similar expression relating $P$, $T$, and $M$. We could continue to use this technique of Jacobians to further transform $(\partial P/\partial T)_{M}$; but an easier method is to expand the equation of state as
\begin{equation}\label{e.eos-logarithmic}
\ln P = \chi_{\rho}\ln\rho + \chi_{T}\ln T,
\end{equation}
where $\chi_{\rho} \equiv (\partial\ln P/\partial\ln \rho)_{T}$, and similarly for $\chi_{T}$.  In an homologous expansion or contraction, the polytropic index stays constant, so that from equations~(\ref{e.LE-density-ratio}) and (\ref{e.LE-PC}) we have
\begin{equation}\label{e.homologous-eos-changes}
\tderiv{\ln P}{\ln R}{M} = -4,\qquad \tderiv{\ln\rho}{\ln R}{M} = -3.
\end{equation}
Using these relations, we have
\begin{eqnarray*}
 \tderiv{\ln T}{\ln R}{M} &=& \chi_{T}^{-1}\left[\tderiv{\ln P}{\ln R}{M} - \chi_{\rho}\tderiv{\ln\rho}{\ln R}{M}\right]\\
   &=& \frac{-4 + 3\chi_{\rho}}{\chi_{T}}.
\end{eqnarray*}
We can use this along with equations~(\ref{e.eos-logarithmic}) and (\ref{e.homologous-eos-changes}) to obtain
\begin{eqnarray}\label{e.dPdT-M}
 \tderiv{\ln P}{\ln T}{M} &=& \tderiv{\ln P}{\ln R}{M}\tderiv{\ln T}{\ln R}{M}^{-1}\nonumber \\
 &=& \frac{4\chi_{T}}{4-3\chi_{\rho}}.
\end{eqnarray}
Inserting eq.~(\ref{e.dPdT-M}) into equation~(\ref{e.homologous-cstar}), we finally get the expression for the gravithermal specific heat under a homologous expansion or contraction,
\begin{equation}\label{e.gravithermal-specific-heat}
	C_{\star} = C_{P}\left[1 - \nablaad\frac{4\chi_{T}}{4 - 3\chi_{\rho}}\right].
\end{equation}
For an ideal gas, $\chi_{T}=\chi_{\rho}=1$ and $\nablaad = 2/5$, so that $C_{\star} < 0$, as required for stability.  As the core becomes degenerate, however, $\chi_{\rho} \to 5/3$ and $\chi_{T}\to 0$; as a result, the addition of heat to the core causes the temperature to rise, causing the rate of heating from nuclear reactions to increase even further.  The ignition of helium in low-mass stars is therefore somewhat unstable and proceeds via ``flashes.''

Eventually, the burning of helium heats the core enough that degeneracy is lifted, and the star settles onto its ``helium main-sequence.''  On an HR diagram, low-mass stars with core helium burning lie on the ``horizontal branch.''

\subsection{The Asymptotic Giant Branch}\label{s.agb}

After exhaustion of helium burning in the core, the star has a semi-degenerate C/O core, surrounded by a He burning shell with a superincumbent H-burning shell.  Burning by these shells adds to the mass of the C/O core. As on the giant branch, the luminosity increases due to the high pressure at the edge of the core, and this high-luminosity produces a deep convective envelope, so that the star again lies along the Hayashi line in the HR diagram. 

Because of the large gravitational acceleration at the edge of the core, the burning shells become thin in radial extent.  This tends to make the burning unstable and leads to \emph{thermal pulses}.
To understand why the burning is unstable, let's revisit equation~(\ref{e.gravithermal-specific-heat}).  This equation still holds for expansion or contraction of a thin layer, with one critical difference: the volume of the shell is $4\pi r_{c}^{2}D$, where $D$ is the thickness of the shell and $r_{c}$ is the core radius, which is fixed. Hence if we expand or contract the shell, keeping the mass in the shell fixed, the logarithmic change in density is
\[ \dif\ln\rho = -\frac{\dif D}{D} = -\frac{\dif R}{D} = -\frac{R}{D}\dif\ln R. \]
Replacing the coefficient 3 of $\chi_{\rho}$ with $R/D$ in equation~(\ref{e.gravithermal-specific-heat}) gives us the specific heat during a homologous expansion or contraction of a shell of thickness $D$,
\begin{equation}\label{e.thin-shell-specific-heat}
 C_{\star,\mathrm{shell}} = C_{P}\left[1-\nablaad\frac{4\chi_{T}}{4-(R/D)\chi_{\rho}}\right].
\end{equation}
Even for an ideal gas, if $D < R/4$, say, then $C_{\star,\mathrm{shell}} > 0$ and the burning in the shell is thermally unstable.

When the burning shells are thin, the temperature, and hence rate of burning, become dependent on the local gravitational acceleration.  Because the underlying core is degenerate, this means the luminosity depends almost entirely on the core mass: an empirical fit is\cite{Paczynski1970Evolution-of-Si}
\begin{equation}\label{e.paczynski}
\frac{L}{\Lsun} = 5.9\ee{4}\left(\frac{M_{c}}{\Msun}-0.52\right).
\end{equation}
The rate at which mass is added to the core is set by the rate at which \hydrogen\ is processed into \helium,
\begin{equation}\label{e.mdot-core}
\frac{\dif M_{c}}{\dif t} = \frac{L}{q}.
\end{equation}
Here $q$ is the mass-specific energy release from the fusion of 4 \hydrogen\ into \helium: $q = 26.72\nsp\MeV/(4\mb) = 6.68\ee{18}\nsp\ergspergram$.  Thus as the core mass grows, the luminosity increases and the rate at which mass is added to the core increases.  The high luminosity drives a strong wind from the stellar envelope.  An empirical fit to the mass loss rate is\cite{Reimers1977On-the-absolute}
\begin{equation}\label{e.wind-loss}
\frac{\dif M}{\dif t} = - 8.0\ee{-13}\nsp\Msun\usp\yr^{-1} \left(\frac{L}{\Lsun}\frac{g_{\sun}}{g}\frac{\Rsun}{R}\right).
\end{equation}
Here $g = GM/R^{2}$ is the gravitational acceleration at the stellar surface.  The coefficient has been increased beyond that in the original formula  to fit the higher mass-loss rates observed from supergiants\cite{Schroder2001The-galactic-ma}.  The envelope is thus consumed at the base by hydrogen and helium burning shells and expelled at the top by a radiative wind. This process ends when the envelope is consumed, leaving behind a degenerate C/O white dwarf that gradually cools.

\newpage
\begin{exercisebox}
This exercise gives an illustration of the physics that sets the white dwarf initial mass function.
Note that the numbers listed here are rather crude.

\begin{enumerate}
\item\label{p.core-mass-growth} Combine equations~(\ref{e.paczynski}) and (\ref{e.mdot-core}) into a differential equation for the core mass as a function of time.  Solve the equation.  Since we don't know the initial core mass $M_{c0}$ at the end of core He burning (other than that we are assuming it is greater than $0.52\nsp\Msun$), let's leave that as a free parameter in the problem. What is the characteristic timescale for the core to increase in mass?

\item Solve equation~(\ref{e.wind-loss}) for the \emph{total} stellar mass as a function of time, assuming the initial total mass at the start of the AGB phase is $M_{0}$.  To make the problem concrete, assume that
 the surface effective temperature on the AGB is fixed at $4000\nsp\K$,  and use the result of problem~\ref{p.core-mass-growth} to get the luminosity as a function of initial core mass, $M_{c0}$.

\item For a star that starts its AGB phase with $M_{0} = 1.0\nsp\Msun$ and $M_{c0}=0.55\nsp\Msun$, what is the final white dwarf mass?
\end{enumerate}
\end{exercisebox}
