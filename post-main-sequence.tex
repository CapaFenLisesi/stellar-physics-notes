% !TEX root = ./notes.tex
\chapter{Post-Main Sequence Evolution}

The consumption of \helium\ is hindered by the lack of stable nuclei with mass numbers $A=5$ and $A=8$.  
The nucleus \beryllium[8] is, however, long-lived by nuclear standards: its decay width is $\Gamma = 68\nsp\eV$, implying a decay timescale $\hbar/\Gamma = 9.7\ee{-17}\nsp\second$.  (For comparison, the lifetime of \lithium[5] is $\sim 10^{-22}\nsp\second$.)  Thus if the reaction $\helium[4]+\helium[4]\to\beryllium[8]$ can proceed quickly enough, a small amount of $\beryllium[8]$ can accumulate allowing the reaction $\beryllium[8]+\helium\to\carbon$ to proceed.

The reaction $2\nsp\helium\to\beryllium[8]$ is endothermic, with $Q = -92\nsp\keV$.  As a result, the peak energy for Coulomb barrier transmission (see the discussion following eq.~[\ref{e.integral}]) must reach
\[ \Epk = \frac{\EG^{1/3}(kT)^{2/3}}{4^{1/3}} = -Q. \]
Substituting $\EG = 979\nsp\keV A (Z_{1}Z_{2})^{2}$ and solving for $T$ gives $T = 1.2\ee{8}\nsp\K$ as the temperature required to build up any substantial amount of \beryllium[8].  To reach such a temperature requires a helium core mass of $\gtrsim 0.45\nsp\Msun$.

Once a sufficient temperature is reached, the reaction $2\nsp\helium[4]\to\beryllium[8]$ comes into equilibrium with the decay, $\beryllium[8]\to2\nsp\helium$.  We can use a Saha-like equation (cf.\ eqns.~[\ref{e.ionization-chem-potential}] and [\ref{e.saha-eqn}]) to get the abundance of \beryllium[8].  Writing
\[ 2\mu_{4} = \mu_{8} - Q \]
and substituting the expression for $\mu$, eq.~(\ref{e.chem-pot-ideal-gas}) gives
\begin{equation}\label{e.n8-n4}
\frac{n_{8}}{n_{4}} = \left(\frac{2\pi\hbar^{2}}{\mb kT}\right)^{3/2}\left(\frac{8}{4^{2}}\right)^{3/2}\frac{X_{4}\rho}{4\mb}\exp\left(-\frac{92\nsp\keV}{kT}\right).
\end{equation}
Here we use $n_{8}$ and $n_{4}$ to mean the number densities of, respectively, \beryllium[8] and \helium; also $X_{4}$ denotes the mass fraction of helium.
Scaling eq.~(\ref{e.n8-n4}) to $\rho = \rho_{5} 10^{5}\nsp\grampercc$ and $T = T_{8}10^{8}\nsp\K$, we obtain
\begin{equation}\label{e.n8-to-n4}
 \frac{n_{8}}{n_{4}} = 2.8\ee{-5} T_{8}^{-3/2}X_{4}\rho_{5}\exp\left(-\frac{10.68}{T_{8}}\right).
\end{equation}
At $X_{4}\rho_{5}=1$ and $T_{8} = 1$, $n_{8}/n_{4} = 6.5\ee{-10}$.  

If the reaction $\beryllium[8]+\helium\to\carbon$ were non-resonant, the reaction rate for this $n_{8}$ would be far too slow to account for the amount of \carbon\ synthesized in stars.  Hoyle proposed, therefore, that there should be an excited state of the \carbon\ nucleus into which the reaction would proceed.  Both \helium\ and \beryllium[8] have spin and parity $J^{\pi} = 0^{+}$; hence for $s$-wave capture (angular momentum $\ell = 0$), the state in \carbon\ should also have $J^{\pi} = 0^{+}$.  What energy should the level have? The $Q$-value for $\beryllium[8]+\helium\to\carbon$ is $7.367\nsp\MeV$; the Gamow energy for this reaction is $\EG = 1.67\ee{5}\nsp\keV$ and hence the peak energy is $\Epk = 146\nsp\keV$, with a width $\Delta = 4(\Epk kT/3)^{1/2} = 82\nsp\keV$.  The proposed level should therefore have an energy within $2\Delta$ of $7.513\nsp\MeV$.

Such a level was indeed detected by Fowler, with $J^{\pi} = 0^{+}$ and $E = 7.654\nsp\MeV$.  Radiative decay from this level is hampered: the ground state also has $J^{\pi} = 0^{+}$, so the decay is forbidden; and the decay to the $J^{\pi}=2^{+}$ state at $4.44\nsp\MeV$ has a decay width of only $\Gamma_{\mathrm{rad}} = 3.67\milli\eV$.  This level's primary decay is indeed back to $\beryllium[8] + \helium$.  If the forward rate is fast enough, however, than a population of \carbon\ in this excited state can accumulate.  The total rate to the ground state would then be $n_{12*}\Gamma_{\mathrm{rad}}/\hbar$, where $n_{12*}$ is the number density of \carbon\ nuclei in the excited state.

To compute $n_{12*}$, we again can use the equation for chemical equilibrium: $\mu_{8} + \mu_{4} = \mu_{12*} - Q_{*}$.  Here $Q_{*} = -287\nsp\keV$ is the difference in energy between the \helium\ and \beryllium[8] nuclei and the energy of \carbon\ \emph{in the excited state.}  Again using eq.~(\ref{e.chem-pot-ideal-gas}) to expand $\mu$, we obtain
\begin{equation}\label{e.n12}
n_{12*} = \frac{n_{Q,12*}}{n_{Q,4}n_{Q,8}} n_{4}n_{8}\exp\left(-\frac{287\nsp\keV}{kT}\right).
\end{equation}
Substituting for $n_{8}$ using equation~(\ref{e.n8-n4}), this becomes
\begin{equation}\label{e.n12-2}
n_{12*} = \left(\frac{2\pi\hbar^{2}}{\mb kT}\right)^{3}\left(\frac{12}{4^{3}}\right)^{3/2} 
	\left(\frac{X_{4}\rho}{4\mb}\right)^{3} \exp\left(-\frac{-397\nsp\keV}{kT}\right).
\end{equation}
Multiplying eq.~(\ref{e.n12-2}) by $\Gamma_{\mathrm{rad}}/\hbar$ and scaling to $\rho_{5}$, $T_{8}$, we obtain the net rate, per unit volume, at which $3\nsp\helium\to\carbon$,
\begin{equation}\label{e.triple-alpha}
 \textrm{rate} = 4.37\ee{31}\nsp\cm^{-3}\usp\second^{-1}\frac{(X_{4}\rho_{5})^{3}}{T_{8}^{3}}\exp\left(-\frac{44.0}{T_{8}}\right).
\end{equation}
Multiplying the rate by $Q = 7.275\nsp\MeV$, the net $Q$-value, gives the volumetric heating rate $\rho\varepsilon_{3\alpha}$, or
\begin{equation}\label{e.triple-alpha-heating}
\varepsilon_{3\alpha} \approx 5.1\ee{21}\nsp\ergs\usp\gram^{-1}\usp\second^{-1}\frac{X_{4}^{3}\rho_{5}^{2}}{T_{8}
^{3}}\exp\left(-\frac{44.0}{T_{8}}\right).
\end{equation}
The temperature exponent is $\dif\ln\varepsilon_{3\alpha}/\dif\ln T = 44.0/T_{8}-3$, that is, $\varepsilon_{3\alpha}\sim T^{41}$ at $T = 10^{8}\nsp\K$.