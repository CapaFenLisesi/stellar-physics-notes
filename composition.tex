% !TEX root = ./notes.tex
\chapter[Composition]{Specifying the Composition of a Multi-Component Plasma}
\label{s.composition}
\newcommand{\mpt}{\ensuremath{m_{\mathrm{p}}}}
\newcommand{\mel}{\ensuremath{m_{\mathrm{e}}}}
\newcommand{\mol}{\unitstyle{mol}}

In this appendix we'll look at how one specifies the composition for a multi-component plasma.  To make things concrete, let's imagine a box containing a mixture of nuclei, of many different isotopes, and electrons.  (To keep things simple, we'll assume complete ionization.)  Each isotope species $i$ has $N_{i}$ nuclei present, and is characterized by charge number $Z_{i}$ and nucleon number $A_{i}$.  Charge neutrality then specifies the number of electrons,
\begin{equation}\label{e.number-e}
N_{\mathrm{e}} = \sum_{i} Z_{i} N_{i}.
\end{equation}
The total mass of the box is
\begin{equation}\label{e.total-mass}
M = \mel N_{\mathrm{e}} + \sum_{i} m_{i}N_{i},
\end{equation}
where $\mel$ and $m_{i}$ are respectively the mass of an electron and a nucleus of species $i$.  Now what is $m_{i}$? Breaking a nucleus $i$ into $Z_{i}$ electrons, $Z_{i}$ protons, and $A_{i}-Z_{i}$ neutrons takes a certain amount of energy, the \emph{binding energy} $B_{i}$.  We can therefore write $m_{i} = Z_{i}\mpt + (A_{i}-Z_{i})\mn - B_{i}/c^{2}$, where $\mpt$ and $\mn$ are respectively the proton and neutron rest masses.

Inserting our expression for $m_{i}$ into equation~(\ref{e.total-mass}), dividing by the volume of the box $V$, and rearranging terms gives us the mass density,
\begin{equation}\label{e.rho}
\rho = \frac{M}{V} = \sum_{i} n_{i}\left[ \left(A_{i}-Z_{i}\right) \mn + Z_{i}\left(\mpt + \mel\right) - B_{i}/c^{2}\right].
\end{equation}
Here $n_{i}$ is the number density of isotope species $i$, and we have used equation~(\ref{e.number-e}) to eliminate $N_{\mathrm{e}}$.  The numbers $n_{i}$ are, of course, fantastically large, so chemists and astronomers define \emph{Avogadro's constant} to be
\begin{equation}\label{e.avogadro-def}
\NA \equiv 6.0221367\ee{23}\nsp\unitstyle{mol}^{-1},
\end{equation}
that is, in one \emph{mole} of anything, there are $\NA$ items. If we multiply and divide the right-hand side of equation~(\ref{e.rho}) by $\NA$, we then have
\begin{equation}\label{e.molar-1}
\rho = \sum_{i} \left(\frac{n_{i}}{\NA}\right) \mathcal{A}_{i},
\end{equation}
where
\begin{equation}\label{e.gm-mol}
\mathcal{A}_{i} = \left[ \left(A_{i}-Z_{i}\right) \mn + Z_{i}\left(\mpt + \mel\right) - B_{i}/c^{2}\right]\times\NA
\end{equation}
is the \emph{gram-molecular weight} of species $i$ with dimensions $[\mathcal{A}]\sim[\gram\cdot\mol^{-1}]$.

Now you may wonder where the numerical value of $\NA$ came from.  It was not pulled out of thin air, but is defined so that 1\usp\mol\ of \carbon\ has a mass of exactly 12\nsp\gram.  In other words, for \carbon\, $\mathcal{A} \equiv A\nsp\gram\usp\mol^{-1}$.  In fact for all nuclei, $\mathcal{A} \approx A \usp\gram\nsp\mol^{-1}$ to better than about 1\%, as demonstrated in Table~\ref{t.gm-mol}.

\begin{table}[htbp]\caption{\label{t.gm-mol}Selected gram-molecular weights.}
\begin{tabular}{r|ccc}
nuclide & $A$ & $\mathcal{A}$ & $(|\mathcal{A}-A|/A) \times 100$\\
\hline
\neutron & 1 & 1.00865 & 0.865\\
\hydrogen & 1 & 1.00783 & 0.783\\
\helium & 4 & 4.00260 & 0.065\\
\oxygen & 16 & 15.99491 & 0..032\\
\silicon & 28 & 27.97693 & 0.082\\
\iron & 56 & 55.93494 & 0.116\\
\hline
\end{tabular}
\end{table}

Because in CGS $\mathcal{A}\approx A$, it is customary to write $\mathcal{A} = A\times (1\usp\gram\nsp\mol^{-1})$, so that equation~(\ref{e.molar-1}) is
\begin{equation}\label{e.molar-2}
\rho = \sum_{i} \left(\frac{n_{i}}{\NA}\times 1\frac{\gram}{\mol}\right) A_{i}.
\end{equation}
This equation is exact if $A$ is now understood to be a real number, but the custom is to just keep it as the nucleon number. Astronomers typically then commit the sin of omission and \emph{redefine} $\NA$ in this context to mean $\NA / (1\nsp\gram\usp\mol^{-1}) = 6.022\ee{23}\nsp\gram^{-1}$. Like all sins, this can lead to grief: you can only get away with this in CGS. I prefer to use the atomic mass unit, defined as $1/12$ the mass of an atom of \carbon, so that $1\amu =  (1\nsp\gram\usp\mol^{-1})/\NA = 1.66054\ee{-24}\nsp\gram$. This puts equation~(\ref{e.molar-2}) into the more obvious form $\rho = \sum n_{i}\times A_{i}\mb$.

With the redefinition of \NA, equation~(\ref{e.molar-2}) can be rewritten as
\begin{equation}\label{e.}
1 = \sum_{i}\left(\frac{n_{i}}{\NA\rho}\right)A_{i} \equiv \sum_{i}Y_{i}A_{i}
\end{equation}
where $Y_{i} \equiv n_{i}/\rho/\NA$ is the \emph{molar fraction}. It is customary to call $Y_{i}A_{i}$ the \emph{mass fraction} $X_{i}$, with $\sum X_{i} = 1$. We can then define the mean atomic mass number,
\begin{equation}\label{e.mean-A}
\bar{A} = \frac{\sum A_{i}Y_{i}}{\sum Y_{i}} = \frac{1}{\sum Y_{i}},
\end{equation}
and mean charge number
\begin{equation}\label{e.mean-Z}
\bar{Z} = \frac{\sum Z_{i}Y_{i}}{\sum Y_{i}} = \bar{A} \sum Z_{i}Y_{i}.
\end{equation}
The molar fraction of electrons is
\begin{equation}\label{e.Ye}
Y_{e} = \sum Z_{i} \frac{n_{i}}{\rho\NA} = \sum Z_{i}Y_{i} = \frac{\bar{Z}}{\bar{A}}.
\end{equation}
In stellar structure work, it is common to use the \emph{mean molecular weight}, defined so that the total number of particles, including electrons, per unit volume is
\begin{equation}\label{e.mean-molecular-weight}
\sum_{i} n_{i} + n_{e} \equiv \frac{\rho\NA}{\mu}.
\end{equation}
Yes, this is still the redefined $\NA$: $\mu$ is dimensionless. From the definition,
\[
\mu = \left(\sum_{i}Y_{i} + Y_{e}\right)^{-1} = \left[ \sum_{i}\left(Z_{i}+1\right)Y_{i} \right]^{-1};
\]
sometimes astronomers also define the mean ion molecular weight, $\mu_{I} = (\sum Y_{i})^{-1}$, and the mean electron weight, $\mu_{e} = Y_{e}^{-1}$.

\section{Exercises}
\begin{enumerate}
\item Consider a gas of \hydrogen\ and \helium\ with molar hydrogen fraction $Y_{\mathrm{H}}$.  Derive expressions for the molar fraction of \helium, $Y_{\mathrm{He}}$, $\bar{A}$, $\bar{Z}$, and $\mu$. What are the numerical value of these quantities for $Y_{\mathrm{H}} = 0.7$, i.~e., solar?
\item Assume that we can describe this plasma as an ideal gas.  What is the sound speed? What is the average kinetic energy of a particle?
\end{enumerate}
