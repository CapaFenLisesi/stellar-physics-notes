% !TEX root = ./notes.tex
\chapter{Neutron Star Cooling}

First, let's get some estimates of the neutron star.  A solar mass has of order $10^{57}$ nucleons.  If we pack those nucleons so that the mean spacing is $\approx 1\nsp\fermi$, then the radius of our object is $\sim 10^{19}\nsp\fermi\sim 10\nsp\km$.  The current best observational constraints put the radius at around 12\nsp\km.  Observed masses (from radio pulsars in binaries) range from around 1.1\nsp\Msun\ up to 1.97\nsp\Msun.

Although the matter in the core of a neutron star is not an ideal gas, we can get some useful estimates by ignoring the nuclear force and imagining that the core consists of cold  neutrons, proton, and electrons is a charge-neutral plasma that is in $\beta$-equilibrium. This translates into three conditions:
\begin{enumerate}
\item Cold: The chemical potentials are just the Fermi energies, $\mu = E_{\mathrm{F}}$, with
\begin{eqnarray}
	E_{\mathrm{F}} &=& \frac{\hbar^{2}}{2m}\left(3\pi^{2}n\right)^{2/3},\qquad\textrm{non-relativistic}\\
	E_{\mathrm{F}} &=& \hbar c\left(3\pi^{2}n\right)^{1/3},\qquad\textrm{relativistic}.
\end{eqnarray}
\item Charge-neutral: $n_{e} = n_{p}$.
\item Beta-equilibrium: The equation $\nt\leftrightarrow \pt+e$ is in equilibrium, so 
$ \mu_{n} = \mu_{p}+\mu_{e}$,
where $Q = (m_{n}-m_{p}-m_{e})c^{2}$
\end{enumerate}

