% !TEX root = ./notes.tex
\chapter{Notes on Radiation Transport}

\newcommand{\unitvector}[1]{\ensuremath{\bvec{\hat{#1}}}}
\newcommand{\unitn}{\unitvector{n}}
\newcommand{\unitk}{\unitvector{k}}
\newcommand{\Ledd}{\ensuremath{L_{\mathrm{Edd}}}}
\newcommand{\Prad}{\ensuremath{P_{\mathrm{rad}}}}
\newcommand{\Pgas}{\ensuremath{P_{\mathrm{gas}}}}


The sun is very opaque.  Were photons able to stream freely, they would exit in $\sim \Rsun/c = 2.0\nsp\second$.  We saw that is instead takes millions of years for the sun to radiate away its stored thermal energy (see eq.~[\ref{e.K-H}]).  As a result, we can regard the sun as a cavity filled with photons with a very slight leakage.  This is the description commonly invoked to describe blackbody radiation, and we expect that in the interior of the sun, the radiation can be described by a photon gas in thermal equilibrium at the ambient temperature.

\section{Description of the Radiation Field}

Consider a cavity containing a gas of photons. In general we can describe the mean number of photons in this cavity as
\begin{equation}\label{e.photon-occupation}
 N = \frac{2}{h^{3}}\int f(p,x)\,\dif^{3}x\,\dif^{3}p.
\end{equation}
Here $f$ is a distribution function; if we are in thermal equilibrium, $f$ is the Bose-Einstein distribution, but our discussion will be more general.  Now consider a small surface on our cavity with area $\dif A$ and unit normal $\unitn$.  The energy incident on this area in a time $\dif t$ having propagation vector along $\unitn$ and propagating into solid angle $\dif\Omega$ is found by integrating equation~(\ref{e.photon-occupation}) over a volume $c\dif t\,\dif A$,
\[
\dif E = \dif A\,c\dif t\,\dif\nu\,\left( \frac{2}{h^{3}}p^{2}\dif p\,\dif\Omega\right)  h\nu \, f.
\]
Since the photon momentum is $p = h\nu/c$, we have
\begin{equation}\label{e.specific-intensity}
I_{\nu} \equiv \frac{\dif E}{\dif t\,\dif A\,\dif\Omega\,\dif\nu} = \frac{2h\nu^{3}}{c^{2}} f.
\end{equation}
This defines the \emph{specific intensity} $I_{\nu}$.  It is easy to show that in the absence of interactions with matter, $I_{\nu}$ is conserved along a ray.  If the photons are in thermal equilibrium, then we can replace $f$ with the Bose-Einstein distribution,
\begin{equation}\label{e.Bnu}
B_{\nu} \equiv \frac{2 h\nu^{3}}{c^{2}} \left[\exp\left(\frac{h\nu}{\kB T}\right)-1\right]^{-1}.
\end{equation}
Here $B_{\nu}$ is the \emph{Planck function}.

The energy density per frequency $u_{\nu}$ can be defined as $\dif E/(c \dif t\,\dif A\,\dif\nu)$; comparing this with the definition of $I_{\nu}$, we see that (for a blackbody)
\begin{equation}\label{e.radiation-energy-density}
u_{\nu} = \frac{1}{c}\int I_{\nu}\,\dif\Omega = \frac{8\pi h\nu^{3}}{c^{3}}\left[\exp\left(\frac{h\nu}{\kB T}\right)-1\right]^{-1}.
\end{equation}
The total energy density can be found by integrating over all frequencies, giving
\[ u = \left[\frac{8\pi^{5}\kB^{4}}{15 h^{3}c^{3}}\right] T^{4}\equiv aT^{4} \]
in agreement with statistical mechanics.

The next quantity to define is the \emph{flux} of energy per unit time, with direction $\unitk$ crossing area $\dif A$ with unit normal $\unitn$, into solid angle $\dif\Omega$, and per frequency interval $\dif\nu$,
\begin{equation}\label{e.radiation-flux}
F_{\nu} = \int I_{\nu} (\unitn\cdot\unitk)\,\dif\Omega.
\end{equation}
If we take our polar angle with respect to $\unitk$, then $(\unitn\cdot\unitk)\,\dif\Omega = \cos\theta\,\sin\theta\,\dif\theta\,\dif \phi$; defining the direction cosine $\mu = \cos\theta$, this becomes 
\[ (\unitn\cdot\unitk)\,\dif\Omega = 2\pi \mu\,\dif\mu, \]
and for blackbody radiation
\[ F = \pi B_{\nu}.\]
If we integrate this over all frequencies, we recover the result,
\[ F = \left(\frac{ac}{4}\right) T^{4} = \sigma_{\mathrm{SB}} T^{4}, \]
where $\sigma_{\mathrm{SB}}$ is the Stefan-Boltzmann constant.

\section{Radiative Diffusion}

\section{Equation of Transfer}

Let's start with the \emph{equation of transfer},
\begin{equation}\label{e.transfer}
\frac{1}{c}\partial_{t}I_{\nu} + \bvec{k}\cdot\grad I_{\nu} = \rho \frac{\varepsilon_{\nu}}{4\pi} - \rho\kappa_{\nu} I_{\nu} + \rho\kappa_{\nu}^{\mathrm{sca}}\phi_{\nu}
\end{equation}
for the specific intensity $I_{\nu}$, defined as the energy radiated per unit area per unit time per frequency per solid angle along a direction $\bvec{k}$. Here $\varepsilon_{\nu}$ is the energy spontaneously emitted per unit frequency per unit time per unit mass: the first term on the right hand side represents the energy added to the beam along a path $\dif s$.  The second term on the right hand side is the energy removed from the beam along a path $\dif s$ with $\kappa_{\nu} = \kappa_{\nu}^{\mathrm{abs}} + \kappa_{\nu}^{\mathrm{sca}}$ being the total opacity (absorption plus scattering). The last term is energy added to beam via scattering.  If the scattering is isotropic, then
\begin{equation}\label{e.isosca}
\phi_{\nu} = \frac{1}{4\pi}\int_{0}^{2\pi}\!\!\!\int_{0}^{\pi} I_{\nu}\,\dif\phi\,\sin\theta\,\dif\theta \equiv J_{\nu},
\end{equation}
where $J_{\nu}$ is the mean intensity: the scattering redistributes the energy over all angles.

\subsection{Moments of the intensity}
It is useful to define \emph{moments} of the specific intensity, which are just weighted angular averages.  For example, integrating $I_{\nu}$ over all angles and dividing by $4\pi$ gives
\begin{equation}\label{e.J}
J_{\nu} \equiv \frac{1}{4\pi}\int_{0}^{2\pi}\!\dif\phi\int_{0}^{\pi}\!\sin\theta\,\dif\theta\, I_{\mu} = \frac{1}{2}\int_{-1}^{1} \!\dif\mu \,I_{\nu}.
\end{equation}
Here $\mu = \cos\theta$. For the first moment, we can multiply $I_{\nu}$ by a unit vector $\bvec{k}$, and then dot that into the unit directional vector and integrate over all directions,
\begin{equation}\label{e.H}
H_{\nu} \equiv \frac{1}{4\pi}\int_{0}^{2\pi}\,\dif\phi\int_{0}^{\pi}\sin\theta\,\dif\theta\, I_{\mu}\,\bvec{k}\cdot\bvec{n} = \frac{1}{2}\int_{-1}^{1} \,\dif\mu \,\mu I_{\nu}.
\end{equation}
Finally, we can multiply $I_{\nu}$ by a tensor $\bvec{k}\bvec{k}$; contracting this along $\bvec{n}$ gives
\begin{equation}\label{e.K}
K_{\nu} \equiv \frac{1}{4\pi}\int_{0}^{2\pi}\,\dif\phi\int_{0}^{\pi}\sin\theta\,\dif\theta\, I_{\mu}\,(\bvec{k}\cdot\bvec{n})^{2} = \frac{1}{2}\int_{-1}^{1} \,\dif\mu \,\mu^{2} I_{\nu}.
\end{equation}
If we further integrate equations~(\ref{e.J})--(\ref{e.K}) over all frequencies, we will obtain expressions for the energy density, flux, and radiation pressure,
\begin{equation}\label{e.thermal}
u = \frac{4\pi}{c}J,\quad F = 4\pi H,\quad P = \frac{4\pi}{c}K.
\end{equation}

\subsection{Radiative equilibrium} 
The emissivity $\varepsilon_{\nu}$ and the opacity $\kappa_{\nu}$ describe how the radiation interacts with matter. A condition of steady-state is that the gas not gain or lose energy to the radiation,
\begin{equation}\label{e.rad-equil}
\int_{0}^{\infty}\! \left(\frac{\varepsilon_{\nu}}{4\pi} - \kappa_{\nu}^{\mathrm{abs}} J_{\nu}\right)\,\dif\nu = 0.
\end{equation}
Now suppose that the level populations of the matter are in thermal equilibrium and can be described by a temperature $T$.  In that case, detailed balance must hold, so that
\begin{equation}\label{e.detail-balance}
\frac{\varepsilon_{\nu}}{4\pi\kappa_{\nu}^{\mathrm{abs}}} = B_{\nu}(T),
\end{equation}
where $B_{\nu}(T)$ is the Planck function. This defines \emph{local thermodynamic equilibrium (LTE)}. If the radiation field is, in addition, described by a Planck function \emph{at the same temperature} then we would have complete thermodynamic equilibrium.


\section{Exercises}
\begin{enumerate}
\item If we regard the sun as a large cavity filled with photons, estimate the total energy stored in the radiation field.  If the sun were to suddenly become completely transparent, what would be the resulting luminosity?
\end{enumerate}
