% !TEX root = ./stellar_notes.tex
\section{Thermodynamical derivatives}
\label{s.thermo-derivatives}

A common task in stellar physics is transforming between different derivatives with respect to different thermodynamical quantities.  For example, you may have expressions for $(\partial \kappa/\partial T)_{\rho}$ and $(\partial \kappa/\partial \rho)_{T}$, but you need $(\partial\kappa/\partial T)_{P}$ and $(\partial\kappa/\partial P)_{T}$.  There is a straightforward way to handle transforming from $(\rho,T)$ space to $(P,T)$ space, and that is using Jacobians.  Despite the utility of this technique, it is not commonly discussed in astrophysical texts\cite{landau80:_statis_physic}.

The \emph{Jacobian} is defined as the determinant of a matrix of partial derivatives,
\begin{eqnarray}
\jac{a}{b}{c}{d} &\equiv& \det\left[
	\begin{array}{lr}\tderiv{a}{c}{d} & \tderiv{a}{d}{c}\\
	\tderiv{b}{c}{d} & \tderiv{b}{d}{c} \end{array}\right] \nonumber \\
 & = & \tderiv{a}{c}{d}\tderiv{b}{d}{c} - \tderiv{a}{d}{c}\tderiv{b}{c}{d}.
 \end{eqnarray}
Because interchanging any the rows (or the columns) causes the determinant to change sign, 
\begin{equation}
\jac{b}{a}{c}{d} = -\jac{a}{b}{c}{d}
\end{equation}
and
\begin{equation}
\jac{a}{b}{d}{c} = -\jac{a}{b}{c}{d}.
\end{equation}
Further,
\begin{equation}
\jac{a}{s}{c}{s} = \tderiv{a}{c}{s}\tderiv{s}{s}{a} - \tderiv{a}{s}{c}\tderiv{s}{c}{s} = \tderiv{a}{c}{s},
\end{equation}
and
\begin{equation}
\jac{a}{b}{a}{b} = \tderiv{a}{a}{b}\tderiv{b}{b}{a} - \tderiv{a}{b}{a}\tderiv{b}{a}{b} = 1.
\end{equation}
Hence we can write thermodynamical derivative in terms of Jacobians, for example,
\begin{equation}
\tderiv{T}{P}{S} = \jac{T}{S}{P}{S}.
\end{equation}
Finally, when multiplying two Jacobians, one can ``cancel'' identical upper and lower parts,
\begin{equation}
\jac{a}{b}{c}{d}\jac{c}{d}{s}{t} =\jac{a}{b}{s}{t},
\end{equation}
as can be readily checked by expanding out both the left and right hand sides.

\subsection{Common thermodynamic derivatives}

Certain thermodynamical derivatives occur commonly in working with fluids. The first set express how the pressure relates to the pair $\rho,T$:
\begin{equation}
\chi_{T}\equiv\frac{T}{P}\tderiv{P}{T}{\rho},\qquad\chi_{\rho}\equiv\frac{\rho}{P}\tderiv{P}{\rho}{T}.
\end{equation}
For a fixed composition the equation of state can always be expanded about a point $P_{0}(\rho_{0},T_{0})$ as
\begin{equation}\label{e.eos}
P = P_{0}\left(\frac{\rho}{\rho_{0}}\right)^{\chi_{\rho}}\left(\frac{T}{T_{0}}\right)^{\chi_{T}},
\end{equation}
or equivalently
\[
\frac{\dif P}{P} = \chi_{\rho}\frac{\dif \rho}{\rho} + \chi_{T}\frac{\dif T}{T}.
\]
Here we are implicitly keeping the composition fixed.

The next set concern how quantities transform under adiabatic changes.  For a general equation of state with fixed composition, define the following:
\begin{eqnarray}
\Gamma_{1} &\equiv& \frac{\rho}{P}\tderiv{P}{\rho}{s} = \tderiv{\ln P}{\ln\rho}{s};\\
\frac{\Gamma_{2}-1}{\Gamma_{2}} &\equiv& \tderiv{\ln T}{\ln P}{s} \equiv \nablaad;\\
\Gamma_{3}-1 &\equiv& \tderiv{\ln T}{\ln \rho}{s}.
\end{eqnarray}
The nomenclature is historical. These quantities are not independent: for example, one can show (see exercise~\ref{ex.gamma-trans}) that
\[
\Gamma_{1} = \left[\chi_{\rho} + \chi_{T}\left(\Gamma_{3}-1\right)\right].
\]
Furthermore,
\begin{eqnarray*}
\Gamma_{3}-1 &=& \frac{P}{\rho T}\frac{\chi_{T}}{c_{\rho}},\\
\frac{\Gamma_{2}-1}{\Gamma_{2}} &=& \frac{\Gamma_{3}-1}{\Gamma_{1}}.
\end{eqnarray*}
Furthermore, one can show that the specific heat at constant pressure is $c_{P} = (\Gamma_{1}/\chi_{\rho})c_{\rho}$.

\subsection{A worked example: derivatives of the opacity}

Here's a simple worked example of how we can use these identities.  Suppose we have an opacity $\kappa(\rho,T)$ that is expressed in terms of density and temperature, but we need the quantity $(\partial \kappa/\partial T)_{P}$. We can express $(\partial\kappa/\partial T)_{P}$ as
\begin{eqnarray}
\tderiv{\kappa}{T}{P} &=& \jac{\kappa}{P}{T}{P}\nonumber\\
 &=& \jac{\kappa}{P}{T}{\rho}\jac{T}{\rho}{T}{P}\nonumber\\
 &=& \tderiv{\rho}{P}{T}\left[\tderiv{\kappa}{T}{\rho}\tderiv{P}{\rho}{T}-\tderiv{\kappa}{\rho}{T}\tderiv{P}{T}{\rho}\right]
 \nonumber\\
 &=& \tderiv{\kappa}{T}{\rho} - \tderiv{\kappa}{\rho}{T}\jac{\rho}{T}{P}{T}\jac{P}{\rho}{T}{\rho}\nonumber\\
 &=& \tderiv{\kappa}{T}{\rho} - \tderiv{\kappa}{\rho}{T} \frac{\chi_{T}}{\chi_{\rho}}\frac{\rho}{T}.
\end{eqnarray}
We have to add to $(\partial \kappa/\partial T)_{\rho}$ a term taht allows for the density to change with temperature at constant pressure.

\begin{exercisebox}
\begin{enumerate}
\item Show that 
\[
 	\left(\frac{\partial T}{\partial P}\right)_{S} 
 	\left(\frac{\partial S}{\partial T}\right)_{P} 
 	\left(\frac{\partial P}{\partial S}\right)_{T} = -1
\]

\item \label{ex.gamma-trans} Show that
\[ 
\tderiv{P}{s}{\rho} = \frac{P}{c_{\rho}}\chi_{T},\qquad \tderiv{P}{\rho}{s} = \frac{P}{\rho}\left[\chi_{\rho} + \chi_{T}\left(\Gamma_{3}-1\right)\right].
\]
\end{enumerate}
\end{exercisebox}
