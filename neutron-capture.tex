% !TEX root = ./notes.tex
\chapter{Production of Heavy Elements}

To explore some features of the r-process, let's make a simple calculation of the abundances of a single isotopic chain of zirconium\sidenote{Why zirconium? No particular reason, other than it is not close to a neutron magic number, so that a liquid-drop mass formula won't be too terrible.}.  Suppose we assume that between any two neutron-rich isotopes of zirconium, \zirconium[A+1] and \zirconium[A], the reactions $(\nt,\gamma)$ and $(\gamma,\nt)$ are in equilibrium.  Show that the ratio of the densities of these two isotopes, denoted by $n_{A+1}$ and $n_{A}$, is given by
\begin{equation}\label{e.lnrat}
	\ln\left(\frac{n_{A+1}}{n_{A}}\right) = \ln\left[\frac{g_{A+1}}{g_{A}}\left(\frac{A+1}{A}\right)^{3/2}\frac{n_{\nt}}{q_{\nt}}\right] + \frac{S(\zirconium[A+1])}{\kB T}.
\end{equation}
Here $S(\zirconium[A+1])$ is the neutron separation energy of \zirconium[A+1] (see eq.~\ref{e.Sn}) and
\[
	\frac{1}{q_{\nt}} = \frac{1}{g_{\nt}}\left(\frac{2\pi\hbar^{2}}{m_{\nt}\kB T}\right)^{3/2}.
\]
Now we'll compute an approximate abundance distribution for \zirconium[96]--\zirconium[116].  Since we are only after a crude approximation, let's set $g_{A+1}/g_{A} = 1$ and $(A+1)/A = 1$ in eq.~(\ref{e.lnrat}). Show that with these approximations
\begin{eqnarray}
	\ln\left(\frac{n_{A}}{n_{96}}\right) &=& \ln\left(\frac{n_{A}}{n_{A-1}}\right)
		+ \ln\left(\frac{n_{A-1}}{n_{A-2}}\right)
		+ \ln\left(\frac{n_{A-2}}{n_{A-3}}\right) + \ldots
		+ \ln\left(\frac{n_{97}}{n_{96}}\right) \nonumber \\
		&=& (A-96)\ln\left(\frac{n_{n}}{q_{n}}\right) + \frac{1}{\kB T}\sum_{i=97}^{A} S(\zirconium[i])
\end{eqnarray}
There is one free parameter, namely $n_{96}$, but this can be eliminated by requiring a normalized distribution,
\[ \sum n_{A} = 1. \]
Compute the normalized abundances for this isotopic chain for a temperature and neutron density $(T = 1.5\ee{9}\nsp\K, n_{n} = 10^{23}\nsp\cm^{-3})$.  You may find it easiest to write a program to do this (please turn in your code). For the separation energy, use the liquid drop model, eq.~(\ref{e.weizacker-fmla}) with coefficients taken from table~\ref{t.liquid-drop-coefficients}.  Make a table of the separation energy and abundance as a function of neutron number. Where does the abundance distribution peak?  If you were to let this distribution $\beta$-decay until a stable nucleus were reached, what would you produce? Redo your calculation for $n_{n} = 10^{21}\nsp\cm^{-3}$ and $n_{n} = 10^{25}\nsp\cm^{-3}$ (same $\kB T$). How does this shift the abundance pattern?
