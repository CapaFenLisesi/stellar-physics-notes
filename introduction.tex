% !TEX root = ./notes.tex
\chapter{Hydrostatic Equilibrium}

To begin our study of stellar structure, we should first consider the star that we know best, our sun.  From planetary orbits and the determination of the gravitational constant $G$, we have the mass; our knowledge of the earth-sun distance and observations gives us the radius; and measurements of the solar radiant flux and spectra give us the luminosity and temperature:
\begin{eqnarray*}
M_{\sun} &=& 1.99\ee{33}\nsp\gram\\
R_{\sun} &=& 6.96\ee{10}\nsp\cm\\
L_{\sun} &=& 3.86\ee{33}\nsp\ergspersecond\\
T_{\mathrm{eff}} &=& 5780\nsp\K.
\end{eqnarray*}
Moreover, radiometric dating of meteorites gives the age of the solar system as 4.6\nsp\Giga\yr.  Finally, the composition of the sun is well known \citep{anders.grevesse:abundances,Asplund2005The-Solar-Chemi}; the most abundant elements, with abundances $A > 8.0$ (here the abundance scale is relative to hydrogen, with $A \equiv \log_{10}[N_{\mathrm{el}}/N_{\mathrm{H}}] + 12$), are H(12.00), He(10.93), N(7.78), O(8.66), and C(8.39).

Another salient feature of our sun is its stability: the power output is remarkably constant, varying by less than 0.1\% over several solar cycle \citep{Willson1991The-suns-lumino,Frohlich2004Solar-radiative}, with inferred changes over 2,000\nsp\yr\ on a similar scale \citep{Frohlich2004Solar-radiative}.  On longer timescales, evidence for liquid water over much of Earth history suggest that the power output of the sun cannot have varied greatly over its life.  The first task, then, is to investigate the mechanical and thermal stability of a self-gravitating fluid.

\section{Fluid equation of motion}\label{s.fluid-introduction}

Over scales that are large compared to the collisional mean free paths between particles, we can treat the fluid as a continuous medium.  That is, we suppose that we can find a scale that is infinitesimal compared to the macroscopic scales, but still much larger than the scales for microscopic interactions. Thus, we can define thermodynamic quantities at a location.

Consider such a macroscopically small volume $V$. Its mass is $M = \int_{V} \rho\nsp\dif V$, where $\rho$ is the mass density.  If $\bvec{u}(\bvec{x},t)$ is the velocity, then the flux of mass into the element is
\[
-\int_{\partial V}\rho\vu\vdot \dif \bvec{S} = \frac{\partial}{\partial t}\int_{V}\rho\nsp\dif V
\]
where the right-hand side follows from mass conservation.  Using Gauss's law to transform the left-hand side into an integral over $V$ and combining terms, we have
\[
\int_{V} \left\{ \frac{\partial\rho}{\partial t} + \divr(\rho\vu)\right\}\dif V = 0.
\]
Since this equation holds for any $V$, the integrand must vanish, and we have our first equation,
\begin{equation}\label{e.mass-conv}
\partial_{t}\rho + \divr(\rho\vu) = 0.
\end{equation}
Our next equation is to get the analog of $\bvec{F} = m\bvec{a}$.  Ignoring viscous effects, the net force on our fluid element (with volume $V$) is due to the pressure over its surface $P$ and the gradient of the gravitational potential $\Phi$:
\[
\int_{V}\rho \frac{\dif^{2}\bvec{r}}{\dif t^{2}}\,\dif V = \int_{V}\bvec{F}\,\dif V =  -\int_{V}\rho\grad\Phi\nsp\dif V - \int_{\partial V}P \nsp\dif \bvec{S}.
\]
Transforming the second integral on the right-hand side to a volume integral, and assuming that $\grad \Phi$ and $\grad P$ vary on macroscopic lengthscales, we arrive at an equation for the acceleration,
\begin{equation}\label{e.accel}
\frac{\dif^{2}\vr}{\dif t^{2}} = -\grad\Phi - \frac{1}{\rho}\grad P.
\end{equation}
where $\vr(t)$ is the position of the particle so that the left-hand side is the acceleration.
Here we must be careful: $\vu(\vx,t)$ refers to velocity of the fluid at a given point in space and a given instance of time, \emph{not} to the velocity of a given particle.  A fluid element can still accelerate even if $\partial_{t}\vu = \bvec{0}$ by virtue of moving a different location. At time $t$ this particle has the velocity
\begin{equation}\label{e.rdot}
\left.\frac{\dif\vr}{\dif t}\right|_{t} = \vu(\vx = \vr|_{t},t)
\end{equation}
where we use the fact that the particle is moving along a streamline of the fluid. At a slightly later time $h$, the particle has moved to a location $\vr(t + h) \approx \vr(t) + h\vu$, and the velocity is now
\begin{equation}\label{e.rdoth}
\left.\frac{\dif\vr}{\dif t}\right|_{t+h} = \vu(\vx = \vr|_{t+h},t+h)\approx \vu + h(\vu\cdot\grad\vu + \partial_{t}\vu),
\end{equation}
where we evaluate the derivatives at time $t$. Subtracting equation~(\ref{e.rdot}) from equation~(\ref{e.rdoth}) and dividing by $h$ gives us the acceleration; inserting this into Newton's law and dividing by volume gives us \emph{Euler's} equation of motion,
\begin{equation}\label{e.euler}
\partial_{t}\vu + \vu\cdot\grad\vu = -\grad \Phi - \frac{1}{\rho}\grad P.
\end{equation}
Equations~(\ref{e.mass-conv}) and (\ref{e.euler}) form the first two equations we need to describe stellar structure.

\section{Estimates of solar properties}

From equations~(\ref{e.mass-conv}) and (\ref{e.euler}) we are in a position to estimate, in an order-of-magnitude sense, many of the stellar properties.  First, let's consider the scale for each term in equation~(\ref{e.euler}),
\begin{center}\begin{tabular}{ccccccc}
$\displaystyle \partial_{t}\vu$ & + &
$\displaystyle  \vu\cdot\grad\vu$ & = &
$\displaystyle -\grad \Phi $ & $-$ & 
$\displaystyle \frac{1}{\rho}\grad P$\\
I & & II & & III & & IV
\end{tabular}
\end{center}
For a ``characteristic'' velocity $U$ and lengthscale $R$, we see that terms I and II are both of order $\sim U^{2}/R$ (the timescale is $R/U$).  For term III, we note that $GM/R^{2} = (GM/R)/R \sim U_{\mathrm{esc}}^{2}/R$, where $U_{\mathrm{esc}}$ is the escape velocity.  Finally, for term IV, $(P/\rho)/R \sim c_{s}^{2}/R$, where $c_{s}$ is the speed of sound.  Hence the typical scales of the terms are
\[
\textrm{I} : \textrm{II} : \textrm{III} : \textrm{IV} \sim U^{2} : U^{2} : U_{\mathrm{esc}}^{2} : c_{s}^{2}
\]
It is clear that the terms on the left-hand side are quite negligible, unless we are dealing with stellar explosions; in this case we must have the two terms on the right-hand side balance, and the star is in hydrostatic balance, 
\begin{equation}\label{e.hydrostatic}
\frac{\dif P}{\dif r} = -\rho \frac{GM}{r^{2}}.
\end{equation}
Note that this does not mean that $\vu$ and $\bvec{a}$ are zero; it simply means that they are not important for establishing the mechanical structure of the star.

\section{A worked example: free-fall collapse}

It's worthwhile to imagine what would happen if we suddenly turned off pressure support in the sun, say by having a demon replace each particle with a non-interacting cold particle. For spherically symmetric collapse, let's follow the motion of an observer on the surface.  The mass interior to the observer is $M = \Msun$, so her equation of motion is
\begin{equation}\label{e.free-fall-eq-motion}
\frac{\dif u}{\dif t} = -\frac{GM}{r(t)^{2}}.
\end{equation}
Multiplying both sides by $u = \dif r/\dif t$ and integrating gives
\[
\frac{1}{2} u^{2} = GM\left(\frac{1}{r} - \frac{1}{R}\right),
\]
where $R = r(t=0)$. Defining $x = r/R$ gives
\begin{equation}\label{e.free-fall-non}
\frac{\dif x}{\dif t} = \left[2 \frac{GM}{R^{3}}\left(\frac{1}{x}-1\right)\right]^{1/2}.
\end{equation}
Now, $GM/R^{3}$ has dimension $[\textrm{time}^{-2}]$; furthermore, $M/R^{3} = 4\pi\bar{\rho}/3$, where $\bar{\rho}$ is the average density at the start of collapse.  (For the sun, $\bar{\rho} = 1.4\nsp\grampercc$, just a bit denser than you.) Hence, we can define the \emph{dynamical timescale} as $t_{\mathrm{dyn}}\equiv (G\bar{\rho})^{-1/2}$.  For the sun, $t_{\mathrm{dyn}}\approx 1\nsp\unitstyle{hr}$.  Defining $\tau = t/t_{\mathrm{dyn}}$ in equation~(\ref{e.free-fall-non}) gives us a math problem,
\[
\frac{\dif x}{\dif\tau} = \left(\frac{8\pi}{3}\right)^{1/2}\left(\frac{1}{x}-1\right)^{1/2}
\]
which can be integrated from $x = 1$ to $x=0$ to give
\[
t_{\mathrm{collapse}} = \left(\frac{3\pi}{32}\right)^{1/2}t_{\mathrm{dyn}} \approx 0.5\nsp\unitstyle{hr}
\]
as the time for the sun to collapse if all pressure support were removed.

\section{Energy considerations}

For a spherically symmetric gaseous body in hydrostatic equilibrium, the mass enclosed by radius $r$ satisfies the differential equation $\dif m/\dif r = 4\pi r^{2}\rho$.  Solving for $\rho$, substituting into the equation for hydrostatic balance, eq.~(\ref{e.hydrostatic}), and rearranging terms gives
\[
4\pi r^{3} \frac{\dif P}{\dif r} = -\frac{Gm(r)}{r} \frac{\dif m}{\dif r}.
\]
Integrating both sides from $r = 0$ to $r = R$, and changing variables on the right hand side from $r$ to $m$ gives
\begin{equation}\label{e.virial-1}
\int_{0}^{R} 4\pi r^{3}\frac{\dif P}{\dif r}\nsp\dif r = -3\int_{V} P\nsp \dif V = -\int_{0}^{M}\frac{Gm}{r(m)}\nsp\dif m = E_{\mathrm{grav}},
\end{equation}
where we integrated the left-hand side by parts, used the fact $P(R) \ll P(0)$, and replaced $4\pi r^{2}\nsp\dif r$ with $\dif V$. Now the pressure is related to the internal thermal (kinetic) energy per unit volume $U$.  For a non-relativistic ideal gas, $P = 2 U/ 3$; for  a relativistic gas, such as photons, $P = U/3$.  Defining $\gamma = (P + U)/U$, we can write the total energy of our gaseous sphere as
\begin{eqnarray}
E &=& E_{\mathrm{th}} + E_{\mathrm{grav}} = \int U \nsp\dif V-3\int P\nsp\dif V \nonumber\\
  &=& \frac{1 -3\left(\gamma - 1\right) }{\gamma - 1} \int P\nsp\dif V = \frac{3(\gamma-1)-1}{3(\gamma-1)} E_{\mathrm{grav}}.
\label{e.total-energy-gas}
\end{eqnarray}
This is the just an application of the virial theorem to our star.

First check: suppose our star is a non-relativistic ideal gas. Then $\gamma = 5/3$ and the total energy is 
\[
E = \frac{1}{2}E_{\mathrm{grav}} < 0.
\]
Our star is bound.  Now suppose that the pressure is dominated by photons. Then $\gamma = 4/3$ and $E = 0$, so we should worry about the stability of very luminous stars!

%What is the thermal energy per particle in the sun?  The mean mass per particle is (see appendix~\ref{s.composition}) is $0.6\mb$.  Hence the total number of particles in the sun is $N = \Msun/(0.6\mb) = 2.0\ee{57}$.  We need to know $r(m)$ in equation~(\ref{e.virial-1}) to get $E_{\mathrm{grav}}$, but we know that to order unity $E_{\mathrm{grav}} \approx G\Msun^{2}/\Rsun$, so the thermal energy per particle is
%\begin{equation}\label{e.energy-per-particle}
%\frac{E_{\mathrm{th}}}{N} \approx \frac{G\Msun\mu\mb}{2\Rsun} \approx 1.9\ee{-9}\nsp\ergs = 1.2\nsp\keV.
%\end{equation}
Now suppose the sun were to slowly contract, such that we can still assume hydrostatic equilibrium.  How long would this take?
The time needed to radiate away the thermal energy defines the \emph{Kelvin-Helmholtz timescale},
\begin{equation}\label{e.K-H}
t_{\mathrm{KH}} \equiv \frac{E_{\mathrm{th}}}{L} \approx \frac{G\Msun^{2}}{2\Rsun L_{\sun}} = 16 \nsp\Mega\yr.
\end{equation}
We have written ``approximately'' because we made the approximation that $E_{\mathrm{grav}}  = -G\Msun^{2}/\Rsun$ (in reality it is closer to $-(3/2) G\Msun^{2}/\Rsun$).
The estimated timescale is much less than the age of the earth, and fossils indicate that the sun has not changed dramatically on this timescale.  Hence there is an energy source needed to maintain the interior in thermal steady-state. The total energy per particle, integrated over the lifetime of the sun, is
\[ \frac{\Delta E}{N} \approx \frac{L_{\sun}\times 4.6\nsp\Giga\yr}{N} \approx 0.2\nsp\MeV. \]
This is much larger than chemical reactions could provide (typical energy scale is $1\nsp\eV$). Hence the sun must be powered by nuclear reactions.

In summary, we've taken the observed gross properties of the sun, the equation of motion for a fluid, and the ideal gas equation of state; from these we've deduced that the sun is in hydrostatic balance, that its interior temperature is of order $10^{7}\nsp\K$, and that it would collapse within 10\nsp\Mega\yr\ if there were no nuclear reactions in its core.  What comes next is fleshing out the detailed physics: the equation of state, energy transport, and nuclear reactions.

\section{Exercises}\label{s.intro-exercises}

\begin{enumerate}
\item What is the mean density of the sun? What is the luminous flux (energy/area/time) at 1~AU? What is the orbital period of a test mass just exterior to the radius of the sun?

\item Consider a planar atmophere, in which $-\grad\Phi = \bvec{g} = -g \bvec{e}_{z}$ with $g$ constant. Thus the equation of hydrostatic equilibrium (eq.~[\ref{e.hydrostatic}]) is
\begin{equation}\label{e.planar-hydrostatic}
\frac{\dif P}{\dif z} = -\rho g.
\end{equation}
Suppose we have an isothermal ideal gas, $P = \rho\kB T/(\mu\mb)$, where $T$ is the temperature, $\kB$ is Boltzmann's constant, and $\mu\mb$ is the mass of particles in the gas ($\mb$ is the atomic mass unit), so that the number of particles per unit volume is $N/V = \rho/(\mu\mb)$.  Show that for such a gas the density decreases as
\[
\rho(z) = \rho(0) \exp\left(-z/H\right)
\]
and find an expression for the \emph{scale height} $H$.  Evaluate $H$ for conditions at sea level on Earth. Does the value make sense? Now evaluate $H$ under conditions appropriate for the solar photosphere; in this case what is $H/R_{\sun}$?

\item Equation~(\ref{e.hydrostatic}) must in general be solved numerically for a real equation of state $P = P(\rho)$, but it is useful to construct a toy model to gain insight.  Suppose the sun has a density profile
\[ \rho(r) = \rho_{0}\left(1-\frac{r}{\Rsun}\right) \]
where $\rho_{0}$ is the central density. Further suppose that the equation of state is that of an ideal gas with mean molecular weight $\mu$.  Find the central density, pressure, and temperature in terms of $\Msun$, $\Rsun$, and $\mu$. How do they compare with the values for a constant density star?  Evaluate them numerically for a solar composition (hydrogen mass fraction of 0.7).  Keeping $M$ and $R$ fixed, what happens to the central temperature if the composition is transformed to pure helium? If the nuclear reaction rate depends on temperature, what would this do the luminosity, in the absence of any other changes?

\item Using equation~(\ref{e.mass-conv}), show that equation~(\ref{e.euler}) can be written as
\begin{equation}\label{e.momentum-conv}
\partial_{t}(\rho u_{i}) + \partial_{j}(\rho u_{i}u_{j}) = -\rho\partial_{i}\Phi - \partial_{i}P,
\end{equation}
where the subscripts $i$ denote components and repeated subscripts are understood to be summed over. Interpret the terms on the left-hand side in terms of conservation of momentum.

\item A side benefit of our argument about the scaling of the terms is that $c_{s}\sim U_{\mathrm{esc}} \sim (G\Msun/\Rsun)^{1/2}$.  Use this to get an estimate of the central temperature of the sun in terms of \Msun\ and \Rsun, assuming the composition is an ideal ionized hydrogen plasma.  What is the numerical value of the temperature?

%\item Now compute the total heat content of the sun, expressed as a function of \Msun\ and \Rsun.  At the current power output of the sun, how long would it take to radiate away this energy?  How does this compare with the known age of the sun?

\item Compute the mean kinetic (thermal) energy per hydrogen nucleus in the sun, and express it in electron volts.

%\item Suppose the sun had a radius of 1 AU before starting collapse. How long would collapse take in that case?

\end{enumerate}

