% !TEX root = ./notes.tex
\chapter[Lagrangian Coordinates]{Transformation to Lagrangian Coordinates}

It is often desirable to use the mass enclosed by a surface of radius $r$,
\begin{equation}\label{e.mass}
	m(r,t) = \int_{0}^{r}\! \rho(r',t) 4\pi r'^{2} \,\dif r',
\end{equation}
as a Lagrangian coordinate.
To do this, differentiate eq.~(\ref{e.mass}) w.r.t.\ $r$,
\[ \partial_{r}m = 4\pi r^{2}\rho, \]
and substitute for $\rho$ in the equation of  continuity (eq.~[\ref{e.mass-conv}]).  The first term becomes
\[ 
	\partial_{t}\rho = \partial_{t}\left(\frac{1}{4\pi r^{2}} \partial_{r} m\right) 
	= \frac{1}{4\pi r^{2}}\partial_{r}(\partial_{t}m),
\]
while the second term becomes
\[
	\frac{1}{4\pi r^{2}}\partial_{r}\left(u\partial_{r}m\right);
\]
the equation of continuity therefore becomes
\begin{equation}\label{e.mod-continuity}
	\frac{1}{4\pi r^{2}} \partial_{r}\left( \partial_{t} m + u\partial_{r} m\right) = 0.
\end{equation}
We can integrate this over $r$ to find that $\partial_{t} m + u\partial_{r} m = f(t)$; since $m(0,t) = 0,\;\forall t$, we must have $f(t) = 0$.  Now $\partial_{t} m + u\partial_{r} m = \dif m/\dif t = 0$, so along a streamline, $m$ is a constant.  We can therefore transform from coordinates $(r,t)$ to $(m,t)$ by setting
\begin{eqnarray}
	\label{e.transformation-rule}
	\left.\frac{\partial}{\partial_{t}}\right|_{r} + u\left.\frac{\partial}{\partial r}\right|_{t} 
	&=& \left.\frac{\partial}{\partial t}\right|_{m} \\
	\label{e.transformation-rule-1}
	\left.\frac{\partial}{\partial r}\right|_{t} &=& 4\pi r^{2}\rho \left.\frac{\partial}{\partial m}\right|_{t}.
\end{eqnarray}
