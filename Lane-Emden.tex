% !TEX root = ./notes.tex
\chapter{Polytropes and the Lane-Emden Equation}

\section{Background}\label{s.LE-background}

A classic problem in stellar evolution is the construction of \emph{polytropic} stellar models.  To understand where the term polytropic comes from, let's first consider an ideal gas in hydrostatic equilibrium.  If the gas is in convective equilibrium then we know it lies along an adiabat.  In that case we have the following relations:
\begin{equation}\label{e.ideal-adiabatic-relations} 
T\rho^{1-\gamma} = \mathrm{const};\quad P\rho^{-\gamma} = \mathrm{const};\quad TP^{(1-\gamma)/\gamma} = \mathrm{const}.
\end{equation}
Here $\gamma = C_{P}/C_{\rho}$ is the ratio of specific heats. Furthermore, we have the equation of state,
\begin{equation}\label{e.ideal-eos}
P = \left(\frac{\NA\kB}{\mu}\right)\rho T,
\end{equation}
where $\mu$ is the mean molecular weight and the quantity in parenthesis is $C_{P}-C_{\rho} = \NA\kB/\mu$.

For the Earth's atmosphere, the condensation of water vapor means that one cannot hold $\dif s = 0$ in a rising plume.  This motivated work in the early 1900's, by Kelvin, Lane, Emden, and others, to consider a more general problem, in which $T\dif s = C$, where $C$ is a constant.  A configuration for which this is true is called a \emph{polytropic configuration}. Along an adiabat, $C = 0$. Substituting into the equation for the first law of thermodynamics, $T\dif s = C_{\rho}\dif T -(P/\rho^{2})\dif \rho$, and using equation~(\ref{e.ideal-eos}), we have
\[
\left(C_{\rho}-C\right)\frac{\dif T}{T} = \left( C_{P} - C_{\rho}\right)\frac{\dif\rho}{\rho},
\]
or $T  \propto  \rho^{(C_{P}-C_{\rho})/(C_{\rho}-C)}$. Comparing this with equation~(\ref{e.ideal-adiabatic-relations}), we can define a \emph{polytropic exponent}, $\gamma' = (C_{P}-C)/(C_{\rho}-C)$. Then equation~(\ref{e.ideal-adiabatic-relations}) holds with $\gamma$ replaced by $\gamma'$.  The advantage of this approximation is that it relates density to pressure so that one can solve the equation of hydrostatic equilibrium without simultaneously having to solve for $T(r)$.

\section{The Lane-Emden Equation and Solution}\label{s.LE-solution}

To use the polytropic equation of state, one makes the \emph{ansatz} that the pressure $P$ is related to the density $\rho$ via
\begin{equation}\label{e.polytope}
P(r) = K\rho^{1+1/n}(r)
\end{equation}
where $n$ and $K$ are constants. Further define the dimensionless variable $\theta$ via
\begin{equation}\label{e.theta-def}
\rho(r) = \rho_{c}\theta^{n}(r),
\end{equation}
where the subscript $c$ denotes the central value at $r=0$. Note that since, 
\[ P(r) \propto \rho \times \rho^{1/n} \propto \rho \theta, \]
the quantity $\theta$ plays the role of a dimensionless temperature for an ideal non-degenerate gas.

Substituting these definitions, eq.~(\ref{e.polytope}) and (\ref{e.theta-def}), into Poisson's equation,
\begin{equation}
\nabla^{2}\Phi = 4\pi G\rho,
\end{equation}
and the equation for hydrostatic equilibrium, 
\begin{equation}
\grad P = -\rho\grad \Phi,
\end{equation}
we obtain the \emph{Lane-Emden} equation for index $n$,
\begin{equation}\label{e.LE}
\xi^{-2} \frac{d}{d\xi}\left(\xi^{2}\frac{d\theta}{d\xi}\right) = -\theta^{n}.
\end{equation}
Here $\xi = r/r_{n}$ is the dimensionless coordinate, and
\begin{equation}
r_{n} = \left(\frac{(n+1)P_{c}}{4\pi G\rho_{c}^{2}}\right)^{1/2}
\end{equation}
is the radial length scale.

For a stellar model, we have the following boundary conditions,
\begin{eqnarray}
\label{e.thetabc}\left.\theta(\xi)\right|_{\xi = 0} &=& 1,\\
\label{e.thetapbc}\left.\theta'(\xi)\right|_{\xi=0} &=& 0.
\end{eqnarray}
From the form of equation~(\ref{e.LE}), it follows that $\theta(-\xi) = \theta(\xi)$, that is, the solution is \emph{even}. A power-series solution to $\theta$ out to order $\xi^{6}$ is
\begin{equation}\label{e.series}
\theta(\xi) = 1 - \frac{1}{6}\xi^{2} + \frac{n}{120}\xi^{4} - \frac{n(8n-5)}{15120}\xi^{6} + \mathcal{O}(\xi^{8})
\end{equation}
Finally, there are analytical solutions for $n = 0$, 1, and 5:
\begin{eqnarray}
\theta_{0}(\xi) &=& 1-\frac{\xi^{2}}{6}\label{e.solution-0}\\
\theta_{1}(\xi) &=& \frac{\sin\xi}{\xi}\label{e.solution-1}\\
\theta_{5}(\xi) &=& \left(\frac{3}{3 + \xi^{2}}\right)^{1/2}.\label{e.solution-5}
\end{eqnarray}
We will use these analytical solutions to verify the ordinary differential equation solver in the project.  The location of the first zero, $\xi_{1},$ is taken as the ``radius'' for the stellar model.  For example, if $n = 0$ (eq.~[\ref{e.solution-0}]), $\xi_{1} = \sqrt{6}$. Note that if $n=5$ there is no root; the configuration extends to $\xi \to \infty$.

\subsection{Exercises}
\begin{enumerate}
\item What is the index $n$ for a constant density star? For an isothermal star? For a star with a degenerate, non-relativistic equation of state? For a star in which radiation pressure dominates?
\end{enumerate}
